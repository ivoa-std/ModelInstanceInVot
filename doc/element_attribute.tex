
The \texttt{ATTRIBUTE} element defines either a class attribute or a collection item, both set with atomic values.
The requirements on
the content (especially \texttt{@ref} and  \texttt{@dmrole}, may differ depending on
the usage:


\begin{enumerate}
\item Child of \texttt{INSTANCE}
 The \texttt{ATTRIBUTE} can be seen as a class attribute;
    it must have a \texttt{@dmrole} XML attribute.

In this case, the \texttt{ATTRIBUTE} must be specified by:
  \begin{itemize} 
      \item \texttt{@ref} - reference to a VOTable \texttt{PARAM}, 
      (not a VOTable \texttt{FIELD})
      \item \texttt{@value} - a literal
      \item  if both are provided; \texttt{@value} serves as the default 
      if the reference cannot be resolved
  \end{itemize}  

  
\item Child of \texttt{COLLECTION} 
In this case the host \texttt{COLLECTION} can be seen as a vector and the \texttt{ATTRIBUTE} as one coordinate of the vector. 
It must have  no \texttt{@dmrole} XML attribute or an empty one.

In this case, the \texttt{ATTRIBUTE} must be specified by:
  \begin{itemize} 
      \item \texttt{@ref} - reference to a VOTable \texttt{PARAM} 
      or \texttt{FIELD}
      \item \texttt{@value} - a literal
      \item if both are provided, \texttt{@value} serves as the default if 
      the reference cannot be resolved
  \end{itemize}  
              
\item Any case :

    The \texttt{ATTRIBUTE} must always have a non empty \texttt{@dmtype} XML attribute.
    If the \texttt{ATTRIBUTE}  is located within a \texttt{GLOBALS}, its value must be set with  a  \texttt{@value} since the \texttt{GLOBALS} block is not connected with any data table.
\end{enumerate}  
    
\begin{lstlisting}[caption={Example of an \texttt{ATTRIBUTE} set with either a column reference or a static value (see \ref{ATTRIBUTE_snippet}). If the column reference cannot be resolved, the attribute will be set with its static value.},language=XML]
<INSTANCE dmtype="cube:Observable">
    <ATTRIBUTE dmrole="cube:DataAxis.dependent" dmtype="ivoa:boolean" value="False"/>
    <INSTANCE dmrole="cube:MeasurementAxis.measure" dmtype="meas:Time"
        <INSTANCE dmrole="meas:Measure.coord" dmtype="coords:MJD"
            <ATTRIBUTE dmrole="coords:MJD.date" dmtype="ivoa:real" ref="IDobstime"/>
            <REFERENCE dmrole="coords:Coordinate.coordSys" dmref="IDtimesys"/>
        </INSTANCE>
    </INSTANCE>
</INSTANCE>
\end{lstlisting}  


\begin{table}[!htbp]
\small
\centering
\begin{tabulary}{\linewidth}{|c|J|}       
       \hline 
            \textbf{Attribute} & 
            \textbf {Role}\\
       \hline         \hline  
            \texttt{@dmrole} & 
            Role of the attribute in the DM\\
        \hline 
            \texttt{@dmtype} & 
            Type of the attribute in the DM\\
        \hline 
            \texttt{@ref} & 
            Reference of the \texttt{FIELD} or \texttt{PARAM} that has to be used to set the 
            \texttt{ATTRIBUTE} value.\\
        \hline 
            \texttt{@value}  &
            Default \texttt{ATTRIBUTE} value. This value is taken if there is no 
            \texttt{@ref} attribue or if \texttt{@ref} cannot be resolved.\\
        \hline 
            \texttt{@unit} & 
            \texttt{ATTRIBUTE} unit. Unit applicable to the attribute value which is given 
            by either \texttt{@value} or \texttt{@ref}. \texttt{@unit} must always 
            be compliant with VOUnit \citep{2014ivoa.spec.0523D}. 
            If the attribute value is set by a resolved \texttt{@ref}, 
            \texttt{@unit} must be compliant with the unit of the referenced
            \texttt{FIELD} or \texttt{PARAM}. The inconsistency handling at 
            this level is beyond the scope of this document. The client can 
            deal with them at his own convenience.\\
        \hline 
            \texttt{@arrayindex} & 
            Index of the native value to be taken to set the \texttt{ATTRIBUTE}. 
            The value must be >= 0.
            Must be ignored if the native value is a single value. 
            An error must be risen if \texttt{@arrayindex} is out of range.
            This attribute is always optional.\\
        \hline 
     \end{tabulary}
     \caption{XML attributes of  \texttt{ATTRIBUTE} .} 
     \label{tbl:attribute-att}
 \end{table}

%mir : I cannot identify clearly the list of usage context . 
%mir: numbers are changed w.r.to the previous list above. 
%mir : should we use a, b, c in the Pattern columns? or ?

\begin{table}[!htbp]
\small
\centering
\begin{tabulary}{\linewidth}{|c|c|c|c|c|J|}
    \hline 
        \textbf{@dmrole} &
        \textbf{@dmtype} &
        \textbf{@ref} &
        \textbf{@value} &
        \textbf{@arrayindex} &
        \textbf{Pattern}\\
    \hline     
    \hline  
        MAND &           
        MAND &           
        OPT &           
        OPT &           
        OPT &   
        (a)\\
    \hline   
        MAND &           
        MAND &           
        NO &           
        MAND &           
        OPT &   
        (b)\\
    \hline  
        NO &           
        MAND &           
        OPT &           
        OPT &           
        OPT &   
        (c) \\
   \hline 
\end{tabulary}
     \caption{Valid attribute patterns for \texttt{ATTRIBUTE}. (a) Valid in a \texttt{TEMPLATES} context.        
        The \texttt{ATTRIBUTE} value must be set with the value of the element referenced by \texttt{@ref} 
       If the \texttt{@ref} can not be resolved and \texttt{@value} is present, \texttt{@value} must be taken. Either \texttt{@ref} or \texttt{@value} must be present or both. (b) This pattern 
        is valid in any context.  (c) is valid in the context of a \texttt{COLLECTION} item.    
        The \texttt{ATTRIBUTE} value must be set with \texttt{@value} as \texttt{ATTRIBUTE} value.} 
     \label{tbl:attribute-pattern}
 \end{table}
