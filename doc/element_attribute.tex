
The \texttt{ATTRIBUTE} element defines either a class attribute or a collection item, both set with atomic values.
The requirements on
the content (especially ref and  \texttt{@dmrole} , may differ depending on
the usage:


\begin{enumerate}
\item Descendant of \texttt{GLOBALS}

In this case, the \texttt{ATTRIBUTE} must be specified by:
  \begin{itemize} 
      \item \texttt{@ref} - reference to a VOTable \texttt{PARAM}, 
      (not a VOTable \texttt{FIELD})
      \item \texttt{@value} - a literal
      \item  if both are provided; \texttt{@value} serves as the default 
      if the reference cannot be resolved
  \end{itemize}  

  
\item Descendant of \texttt{TEMPLATES} 

In this case, the \texttt{ATTRIBUTE} must be specified by:
  \begin{itemize} 
      \item \texttt{@ref} - reference to a VOTable \texttt{PARAM} 
      or \texttt{FIELD}
      \item \texttt{@value} - a literal
      \item if both are provided, \texttt{@value} serves as the default if 
      the reference cannot be resolved
  \end{itemize}  

\item Child of \texttt{COLLECTION} 
    The \texttt{ATTRIBUTE} can be seen as a vector coordinate, 
    it must have  no \texttt{@dmrole} xml attribute or an empty one
    
\item Child of \texttt{INSTANCE}  
    The \texttt{ATTRIBUTE} can be seen as a class attribute, 
    it must have a \texttt{@dmrole} xml attribute
           
\item Any case:     
    The \texttt{ATTRIBUTE} must always have a non empty \texttt{@dmtype} xml attribute.
\end{enumerate}  
    
    
\begin{lstlisting}[frame=single,caption={ATTRIBUTE examples},style=XML,basicstyle=\tiny]
<dm-mapping:INSTANCE dmtype="model:Thing">
    <dm-mapping:ATTRIBUTE dmrole="model:Thing.f" dmtype="model.Alpha" value="xyz"/>		
    <dm-mapping:INSTANCE dmrole="model:Thing.pos" dmtype="adhoc:Position">
        <dm-mapping:ATTRIBUTE dmrole="adhoc:Position.longitude" dmtype="ivoa:real" 
                        ref="_eqpos" arrayindex="0"/>
        <dm-mapping:ATTRIBUTE dmrole="adhoc:Position.latitude" dmtype="ivoa:real" 
                        ref="_eqpos" arrayindex="1"/>
    </dm-mapping:INSTANCE>
</dm-mapping:INSTANCE>
\end{lstlisting}  


\begin{table}[!htbp]
\small
\centering
\begin{tabulary}{\linewidth}{|c|J|}       
       \hline 
            \textbf{Attribute} & 
            \textbf {Role}\\
       \hline         \hline  
            \texttt{@dmrole} & 
            Role of the attribute in the DM\\
        \hline 
            \texttt{@dmtype} & 
            Type of the attribute in the DM\\
        \hline 
            \texttt{@ref} & 
            Reference of the \texttt{FIELD} that has to be used to set the 
            \texttt{ATTRIBUTE} value.\\
        \hline 
            \texttt{@value}  &
            Default \texttt{ATTRIBUTE} value. This value is taken if there is no 
            \texttt{@ref} attribue or if \texttt{@ref} cannot be resolved.\\
        \hline 
            \texttt{@unit} & 
            \texttt{ATTRIBUTE} unit. This is the unit in which the native value must be 
            converted to be complient with the model. This attribute is always optional.\\
        \hline 
            \texttt{@arrayindex} & 
            Index of the native value to be taken to set the \texttt{ATTRIBUTE}. 
            The value must be >= 0.
            Must be ignored if the native value is a single value. 
            An error must be risen if \texttt{@arrayindex} is out of range.
            This attribute is always optional.\\
        \hline 
     \end{tabulary}
     \caption{\texttt{ATTRIBUTE} attributes} 
     \label{tbl:attribute-att}
 \end{table}

\begin{table}[!htbp]
\small
\centering
\begin{tabulary}{\linewidth}{|c|c|c|c|c|J|}
    \hline 
        \textbf{@dmrole} &
        \textbf{@dmtype} &
        \textbf{@ref} &
        \textbf{@value} &
        \textbf{@arrayindex} &
        \textbf{Pattern}\\
    \hline     
    \hline  
        MAND &           
        MAND &           
        OPT &           
        OPT &           
        OPT &   
        1 \\
    \hline   
        MAND &           
        MAND &           
        NO &           
        MAND &           
        OPT &   
        2\\
    \hline  
        NO &           
        MAND &           
        OPT &           
        OPT &           
        OPT &   
        3 \\
   \hline 
\end{tabulary}
     \caption{Valid attribute patterns for \texttt{ATTRIBUTE}. (1) Valid in a \texttt{TEMPLATES} context.        
        The \texttt{ATTRIBUTE} value must be set with the value of the element referenced by \texttt{@ref} 
       If the \texttt{@ref} can not be resolved and \texttt{@value} is present, \texttt{@value} must be taken. Either \texttt{@ref} or \texttt{@value} must be present or both (2) This pattern 
        is valid in a any context.  (3) is valid in the context of a \texttt{COLLECTION} item.    
        The \texttt{ATTRIBUTE} value must be set with \texttt{@value}         as \texttt{ATTRIBUTE} value} 
     \label{tbl:attribute-pattern}
 \end{table}
