MIVOT adds a new feature to the VOTable format which allows a new way to consume data. 
The standard was designed so that this new functionality could be ignored by regular clients, 
as services are not required to provide annotated data anyway. 
This flexibility requires a standard way to declare the availability of annotated data. 
The registration of this capability must be applicable to any service running any protocol 
that possibly serves data as VOTables.

In this context it is important to be able inform clients in advance that a service can deliver annotated data.
We identified 2 ways to do this:
\begin{itemize}
    \item Defining a specific mime-type extension allowing clients to tell the server 
          that annotated responses are accepted if available. 
          This has been exercised with the XTapDB (\url{https://xcatdb.unistra.fr/xtapdb/}) service which understands 
          the \texttt{application/mango} ad-hoc accepted mime type as a request for mapping the query results on the Mango data model 
          (still a draft at the time of writing.). The limitation if this approach is that one mime-type has to 
          be defined per model or if we do not want to specify any specific model, e.g. \texttt{mime-type:application/mivot}
          clients will have to parse the VOTable to discover which models are mapped.
    \item The other option is to define a new IVOA Registry capability (see VOResource, \cite{2018ivoa.spec.0625P}) that 
    	  would list the mapped models and that could be appended to the service capabilities. 
\end{itemize}

It is to be noted that these 2 options can be mixed together.
The definitions of both capability and mime type are out of the scope of this standard which is focused on the syntax.

