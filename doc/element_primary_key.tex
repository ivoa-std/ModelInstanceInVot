The \texttt{PRIMARY\_KEY} element allows to set an identification key to an \texttt{INSTANCE}  The primary keys are only used in the context of \texttt{REFERENCE}  using \texttt{FOREIGN\_KEY} 
A primary key can be either static or dynamic.

\begin{itemize}
    \item Static: the key value is given by the \texttt{@value} attribute.
    \item Dynamic: the key value is given by the value of the field referenced by \texttt{@ref}  
    This pattern is only valid if the \texttt{INSTANCE} is within a \texttt{TEMPLATES}. 
\end{itemize}

The type of the key must always be specified by the \texttt{@dmtype} attribute. 

\begin{lstlisting}[caption={The \texttt{INSTANCE} is identified within a \texttt{COLLECTION} by the \texttt{PRIMARY\_KEY} value (see line~\ref{PRIMARY_KEY_snippet} in Appendix~\ref{appendix_A}).},language=XML]
<INSTANCE dmtype="ds:experiment.ObsDataset">
    <PRIMARY_KEY dmtype="ivoa:string" value="5813181197970338560"/>
    ...
</INSTANCE>
\end{lstlisting}

\begin{table}[!htbp]
\small
\centering
\begin{tabulary}{\linewidth}{|c|J|}       
       \hline 
            \textbf{Attribute} & 
            \textbf {Role}\\
       \hline         \hline  
            \texttt{@ref} &
            ID of the FIELD used as primary key \\
        \hline 
            \texttt{@dmtype} & 
            Type of the key \\
        \hline 
            \texttt{@value} & 
            Literal key value. Used when the key relates to a \texttt{COLLECTION} in the \texttt{GLOBALS} \\
        \hline 
     \end{tabulary}
     \caption{\texttt{PRIMARY\_KEY} attributes.} 
     \label{tbl:primarykey-att}
 \end{table}

\begin{table}[!htbp]
\small
\centering
\begin{tabulary}{\linewidth}{|c|c|c|J|}
    \hline 
        \textbf{@ref} &
        \textbf{@dmtype} &
        \textbf{@value} &
        \textbf{Pattern}\\
    \hline      \hline  
        MAND &           
        MAND &           
        NO &           
        The FIELD referenced by \texttt{@ref} is a primary key. This pattern is used within a \texttt{TEMPLATES} \\
    \hline     
        NO &           
        MAND &           
        MAND &           
        \texttt{@value} gives the key value. This pattern is used to set a primary key to a \texttt{COLLECTION}\\
   \hline 
\end{tabulary}
     \caption{Valid attribute patterns for  \texttt{PRIMARY\_KEY}.}
     \label{tbl:primarykey-pattern}
\end{table}
