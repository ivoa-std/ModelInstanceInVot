The connection between data and models was initially meant to be handle by UTypes.
UTypes are path-like strings (m:a.b.c) identifying model leaves. There were no common way to build UTypes; each model (e.g. Characterization, Obscore)  came with their specific lists of UTypes. 

In the VOTable context, UTypes are set as \texttt{FIELD} attributes, allowing so to connect particular columns to model leave. This mapping strategy fits very well with the VOTable schema. There is no need to introduce new XML elements. There are however major limitations that have been discussed many times:

\begin{itemize}
  \item UTypes are pointers from \texttt{FIELD} or \texttt{PARAM} towards the 
  model structure. It then no possible to have one  \texttt{FIELD} or
  \texttt{PARAM} mapped playing different roles.
  \item not intended to be parsable, so only identifies what the leaf 
  FIELD/PARAM is, not its context.
  \item not reusable, the same element used in different contexts/models 
  have different UTypes; which hinders interoperability
  \item constrained to single-table serializations. All model elements must 
  be contained within the same VOTable TABLE.
  \item does not support annotating the VOTable content to multiple models 
  (as TimeSeries or Catalog/Source )
\end{itemize}

The UType approach could have been develop to overcome these drawbacks, but it has been decided to move a forward a solution closer to the VODML concepts. UTypes are still there in many legacy VOtables  but the news VO models won't use them anymore.
