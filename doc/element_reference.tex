INSTANCE reference that can be used with the same patterns as for \texttt{INSTANCE} .
There are different uses for the \texttt{REFERENCE} 

\begin{itemize}
    \item Static reference: the element has a \texttt{@dmref} attribute that matches the \texttt{@dmid} attribute of the referenced \texttt{INSTANCE} 
    \item Dynamic reference: The element has a \texttt{@sourceref} attribute identifying  the table where to fetch the referenced column. 
    
             In this case, \texttt{REFERENCE} must be located in a \texttt{TEMPLATES} and it must have one or more \texttt{FOREIGN\_KEY} children. 
             If the referenced table contains several \texttt{INSTANCE} with a \texttt{PRIMARY\_KEY}  the first match must be taken by default.
\end{itemize}

\begin{lstlisting}[caption={Simple \texttt{REFERENCE}, to be replaced with the \texttt{INSTANCE} having \texttt{@dmid} \_target1.},language=XML]
<REFERENCE dmrole="_role" dmref="_target1" />
\end{lstlisting}

\begin{lstlisting}[caption={Dynamic \texttt{REFERENCE}, to be replaced with the \texttt{INSTANCE} of the table of collection \_target1 and having a \texttt{PRIMARY\_KEY} matching the value of column  \_col1. This pattern is valid in the context of a TEMPLATES.},language=XML]
<REFERENCE dmrole="_role" sourceref="_target1">
    <FOREIGN_KEY ref="_col1" />
</REFERENCE>
\end{lstlisting}

\begin{table}[!htbp]
\small
\centering
\begin{tabulary}{\linewidth}{|c|J|}       
       \hline 
            \textbf{Attribute} & 
            \textbf {Role}\\
       \hline         \hline  
            \texttt{@dmrole} & 
            Role of the referenced instance or collection in the DM \\
        \hline 
            \texttt{@sourceref}  &
            \texttt{@dmid} of the \texttt{COLLECTION} to be joined with in case of using a \texttt{FOREIGN\_KEY} \\
        \hline 
            \texttt{@dmref} & 
            \texttt{@dmid} of the referenced instance or collection\\
        \hline 
     \end{tabulary}
     \caption{\texttt{REFERENCE} attributes.} 
     \label{tbl:reference-att}
 \end{table}

\begin{table}[!htbp]
\small
\centering
\begin{tabulary}{\linewidth}{|c |c |c|J|}
    \hline 
        \textbf{Element} &
        \textbf{Position} &
        \textbf{Occurs} &
        \\
    \hline      \hline  
        \texttt{FOREIGN\_KEY}  &        
        First &           
        0-* &
        Foreign key to be used to resolve a dynamic reference.\\
    \hline 
\end{tabulary}
     \caption{Allowed children for \texttt{REFERENCE}.} 
     \label{tbl:reference-children}
\end{table}


\begin{table}[!htbp]
\small
\centering
\begin{tabulary}{\linewidth}{|c|c|c|J|}
    \hline 
        \textbf{@dmrole} &
        \textbf{@sourceref} &
        \textbf{@dmref} &
        \textbf{Pattern}\\
    \hline      \hline  
        MAND &           
        MAND &           
        NO &           
        This is the \texttt{FOREIGN\_KEY} pattern \texttt{@sourceref} gives the  \texttt{@dmid} of the \texttt{COLLECTION} to be joined with. In this case \texttt{REFERENCE} must have at least one \texttt{FOREIGN\_KEY} child and the joined \texttt{COLLECTION} must have a \texttt{PRIMARY\_KEY}\\
    \hline   
        MAND &           
        NO &           
        MAND &           
        Simple reference to either an \texttt{INSTANCE} or \texttt{COLLECTION}, usually searched in the \texttt{GLOBALS}\\
   \hline 
\end{tabulary}
     \caption{Valid attribute patterns for  \texttt{REFERENCE}.}
     \label{tbl:reference-pattern}
\end{table}

