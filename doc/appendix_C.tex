
In snippet \ref{app:join-pattern-1}, the \texttt{GLOBALS} collection having \texttt{cube:SparseCube.data} as role is populated with  instances of another \texttt{TEMPLATES} (or global \texttt{COLLECTION})
\begin{itemize}
  \item The joined \texttt{TEMPLATES} is this containing the  \texttt{INSTANCE \texttt{@dmid} \_ts\_data}
  \item The joined items are those matching the condition  \texttt{ Results[\_band]='G'}
\end{itemize}

\begin{lstlisting}[frame=single,label={app:join-pattern-1},caption={Joining a global \texttt{COLLECTION} with a \texttt{TEMPLATES}  identified by a \texttt{@dmid} \texttt{@dmref} pair},style=XML,basicstyle=\tiny]
<dm-mapping:VODML>
  ....
  <dm-mapping:GLOBALS>
    <dm-mapping:COLLECTION dmid="_SparseCubes">
      <dm-mapping:INSTANCE dmid="_TimeSeries" dmrole="" dmtype="cube:SparseCube">
        <dm-mapping:REFERENCE dmrole="cube:DataProduct.dataset" dmref="_ds1" />
        
        <dm-mapping:COLLECTION dmrole="cube:SparseCube.data">
          <dm-mapping:JOIN dmref="_ts_data">
            <dm-mapping:WHERE value="G" primarykey="_band" />
          </dm-mapping:JOIN>
        </dm-mapping:COLLECTION>
        
      </dm-mapping:INSTANCE>
    </dm-mapping:COLLECTION>
  </dm-mapping:GLOBALS>

  <dm-mapping:TEMPLATES tableref="Results">
    <dm-mapping:INSTANCE dmid="_ts_data" dmtype="cube:NDPoint">
      ....
      ....
    </dm-mapping:INSTANCE>
  </dm-mapping:TEMPLATES>
</dm-mapping:VODML>
\end{lstlisting}  

Listing \ref{app:join-pattern-2} below shows up another way to map the same join but by using a \texttt{@sourceref}.

\begin{lstlisting}[frame=single,label={app:join-pattern-2},caption={Joining a global \texttt{COLLECTION} with a \texttt{TEMPLATES}  identified by a @sourceref},style=XML,basicstyle=\tiny]
<dm-mapping:VODML>
  ....
  <dm-mapping:GLOBALS>
    <dm-mapping:COLLECTION dmid="_SparseCubes">
      <dm-mapping:INSTANCE dmid="_TimeSeries" dmrole="" dmtype="cube:SparseCube">
        <dm-mapping:REFERENCE dmrole="cube:DataProduct.dataset" dmref="_ds1" />
        
        <dm-mapping:COLLECTION dmrole="cube:SparseCube.data">
          <dm-mapping:JOIN sourceref="Results">
            <dm-mapping:WHERE value="G" primarykey="_band" />
          </dm-mapping:JOIN>
        </dm-mapping:COLLECTION>
        
      </dm-mapping:INSTANCE>
    </dm-mapping:COLLECTION>
  </dm-mapping:GLOBALS>

  <dm-mapping:TEMPLATES tableref="Results">
    <dm-mapping:INSTANCE dmid="_ts_data" dmtype="cube:NDPoint">
      ....
      ....
    </dm-mapping:INSTANCE>
  </dm-mapping:TEMPLATES>
</dm-mapping:VODML>
\end{lstlisting}  

This pattern works since the joined \texttt{@TEMPLATES} has only one child. 
If it had more than one children, the mapped instances would have to be identified by  \texttt{@dmid \texttt{@dmref} pairs} as shown in listing \ref{app:join-pattern-3}

\begin{lstlisting}[frame=single,label={app:join-pattern-3},caption={Joining a \texttt{TEMPLATES} with a global \texttt{COLLECTION} identified by both \texttt{@sourceref} and \texttt{@dmid} \texttt{@dmref} pairs},style=XML,basicstyle=\tiny]
<dm-mapping:VODML>
  ....
  <dm-mapping:GLOBALS>
    <dm-mapping:COLLECTION dmid="_SparseCubes">
      <dm-mapping:INSTANCE dmid="_TimeSeries" dmrole="" dmtype="cube:SparseCube">
        <dm-mapping:REFERENCE dmrole="cube:DataProduct.dataset" dmref="_ds1" />
        
        <dm-mapping:COLLECTION dmrole="cube:SparseCube.data">
          <dm-mapping:JOIN sourceref="Results" dmref="_ts_data">
            <dm-mapping:WHERE value="G" primarykey="_band" />
          </dm-mapping:JOIN>
        </dm-mapping:COLLECTION>
        
      </dm-mapping:INSTANCE>
    </dm-mapping:COLLECTION>
  </dm-mapping:GLOBALS>

  <dm-mapping:TEMPLATES tableref="Results">
    <dm-mapping:INSTANCE dmid="_ts_data" dmtype="cube:NDPoint">
      ....
    </dm-mapping:INSTANCE>
    <dm-mapping:INSTANCE dmid="_other_data" dmtype="cube:OtherType">
      ....
    </dm-mapping:INSTANCE>
  </dm-mapping:TEMPLATES>
</dm-mapping:VODML>
\end{lstlisting}  

In the example \ref{app:join-pattern-4}, the collection having \texttt{cube:SparseCube.data} as role is populated with  instances of another \texttt{TEMPLATES}
\begin{itemize}
  \item The joined \texttt{TEMPLATES} is this containing the  \texttt{INSTANCE \texttt{@dmid} \_ts\_data}
  \item The joined items are those matching the condition  \texttt{\_PKTable[\_pksrcid]=Results[\_srcid] AND  \_PKTable[\_PKband]=Results[\_band]}
\end{itemize}


\begin{lstlisting}[frame=single,label={app:join-pattern-4},caption={Joining two \texttt{TEMPLATES} together with \texttt{@dmid} \texttt{@dmref} pairs},style=XML,basicstyle=\tiny]
<dm-mapping:VODML>
  ....
  <dm-mapping:GLOBALS>
  ....
  ....
  </dm-mapping:GLOBALS>

  <dm-mapping:TEMPLATES tableref="_PKTable">
    <dm-mapping:INSTANCE dmid="_TimeSeries" dmrole="" dmtype="cube:SparseCube">
      <dm-mapping:COLLECTION dmrole="cube:SparseCube.data">
        <dm-mapping:JOIN dmref="_ts_data">
          <dm-mapping:WHERE foreignkey="_srcid" primarykey="_pksrcid" />
          <dm-mapping:WHERE foreignkey="_band" primarykey="_pkband" />
        </dm-mapping:JOIN>
      </dm-mapping:COLLECTION>
    </dm-mapping:INSTANCE>
  </dm-mapping:TEMPLATES>

  <dm-mapping:TEMPLATES tableref="Results">
    <dm-mapping:INSTANCE dmid="_ts_data" dmtype="cube:NDPoint">
      ....
      ....
    </dm-mapping:INSTANCE>
  </dm-mapping:TEMPLATES>
</dm-mapping:VODML>
\end{lstlisting}  

