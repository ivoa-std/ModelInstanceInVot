
The first purpose of a model is to provide, for a particular domain, a formal description of the quantities relevant to that domain and how they relate to each other.
The second purpose is to facilitate the interoperability between  different stakeholders involved in the domain. The interoperability consists in arranging exchanged data 
so that any client can understand it without being concerned with its origin thus allowing the same code to process and compare data coming from different sources.  
In other words, the more faithful the data representation is to the model, the better the interoperability is.
The challenge for the VO is to design the best way to map various data on standard models while respecting to the existing frameworks.

We could imagine to develop a new data container specific for that purpose but any solution that ignores the existing assets would be utterly useless and would have no chance of being accepted.
The model mapping solution must be based on VOTables since VOTable  \citep{2019ivoa.spec.1021O} is the standard data container for the VO.
Unfortunately, if the VOTables schema allows a very good description of tabular data, it is not well suited to rendering views of complex data models.
In the case of simple DAL protocols, this limitation has been worked around by specifying the meta-data that must be present in the query responses; the model mapping is thereby defined by the standard.
This approach is no longer applicable for complex models such as CubeDM or Provenance for which the data tree may vary from one instance to another.

The basis on which the standard is designed is threefold 1) the data content is apriori unknown (e.g. TAP response), it has 2) to be mapped on models that are not defined yet and 3) must be fully contained within a VOTable context.
It must allow to bind native data with a model in a way that model-aware software can see it as 
model instances while maintaining the possibility to access them in their original layout.

The VOTable schema supports an XML attribute for this purpose (UType) that partially meets this function. 
UTypes are path-like strings (\texttt{m:a.b.c}) identifying model leaves. As there is no common rule to build UTypes; each model (e.g. Characterization, Obscore)  must come with its specific lists of UTypes. 

When used as \texttt{FIELD} attributes, UTypes allow one to connect particular columns to model leaves. This mapping method fits very well with the VOTable schema since it does not require any specific XML elements. It has however a number of limitations that have been discussed many times:

\begin{itemize}
  \item UTypes are pointers from \texttt{FIELD} or \texttt{PARAM} towards the 
  model structure. The is no possibility for one  \texttt{FIELD} or
  \texttt{PARAM} to play multiple roles.
  \item UTypes are not intended to be parsable, so they only identify what the leaf 
  \texttt{FIELD}/\texttt{PARAM} is, not its context.
  \item UTypes are not reusable, the same element used in different contexts or models 
  have different UTypes; which hinders interoperability
  \item UTypes constrain to single-table serializations. All model elements must 
  be contained within the same VOTable TABLE.
  \item UTypes do not support annotating the VOTable content to multiple models 
  (as TimeSeries or Catalog/Source)
\end{itemize}

The UType approach could have been developed to overcome these drawbacks, but it has been decided to move forward a solution closer to the \texttt{VODML} \citep{2018ivoa.spec.0910L} concepts. 
 \texttt{VODML} is a meta-model that gives a standard way to express VO models and to make them machine-readable.
In \texttt{VODML},  model leaves are no longer identified by simple strings like UTypes do, but by the role they play in a given location in the model hierarchy.
As a consequence, any annotation mechanism based on \texttt{VODML} and preserving the model hierarchy will be able  to provide a data representation faithful to that model.

The basis of MIVOT is to insert into the VOTable, an XML block complying with the 
model structure and containing references to the actual data.
These blocks are designed in such a way that a model-aware client can build  model instances by copying that structure and by resolving the references. Model-unaware clients can just ignore them. 

%MOVED from section2

%The meta-data currently supported by the  VOTable standard gives a very good description of the table field content. 
%There is however not suitable way to specify the exact role played by a given quantity in a particular context yet.

%There is also no way to describe how different quantities or different data tables relate to each other either.
%UTypes have been proposed to fill these gaps, but the difficulty to define a standard method to build them in the general case of non-hierachical models caused this approach not to be used.
%Actually, both role specifications and quantity associations are parts of the modelling effort and our need is a bridge between model leaves and data columns that allows users to get a model view on data tables. (%NFB Hummm. Is that only for leaves ? Or also for higher levels ?)
%This is essential to get a good  understanding of complex data.


