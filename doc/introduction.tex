
The first purpose of a model is to provide, for a particular domain, a formal description of the quantities relevant to that domain and how they relate to each other.
The second purpose is to facilitate the interoperability between  different stakeholders involved in the domain. The interoperability consists in arranging exchanged data 
so that any client can understand it without being concerned with its origin thus allowing the same code to process and compare data coming from different archives.  
In other words, the more faithful is the data representation towards the data model, the better is the interoperability.
The challenge for the VO ecosystem is to design the best way to map various data on standard models while respecting the existing frameworks, protocols and data formats.

We could imagine to develop a new data container specific for that purpose but any solution that ignores the existing assets would be utterly useless and would have no chance of being accepted.
The model mapping solution must be based on VOTables since VOTable  \citep{2019ivoa.spec.1021O} is the standard data container within the VO.
Unfortunately, if the VOTable schema allows a very good description of tabular data, it is not well suited for rendering views of complex data models.
In the case of simple DAL protocols, this limitation has been worked around by specifying the metadata that must be present in the query responses; the model mapping is thereby defined by the standard.

This approach is no longer applicable for complex models such as Cube or Provenance for which the data tree may vary from one instance to another.

The basis on which the standard is designed is threefold 1) the data content is apriori unknown (e.g. TAP response), 2) it has to be mapped on models that are not defined yet and 3) the mapping block must be fully integrated within a VOTable context.
It must also allow to bind native data with a model in a way that model-aware software can see it as model instances while maintaining the possibility to access them in their original serialization.

The VOTable schema supports an XML attribute for this purpose, a UType, that partially meets this function. 
UTypes are path-like strings (\texttt{m:a.b.c}) identifying model leaves. As there is no common rule to build UTypes,  each data model (e.g. Characterization, Obscore)  must come with its specific lists of UTypes. 

When used as \texttt{FIELD} attributes, UTypes allow one to connect particular columns to model leaves. This mapping method fits very well with the VOTable schema since it does not require any specific XML elements. It has however a number of limitations that have been discussed many times:

\begin{itemize}
  \item UTypes are labels attached to  a \texttt{FIELD} or \texttt{PARAM} that tells its role in the 
  model structure, typically a non cyclic graph. 
   \item There is no possibility for one  \texttt{FIELD} or \texttt{PARAM} to play multiple roles.
  \item UTypes are not intended to be parsed, so they only identify which role is played by a \texttt{FIELD}/\texttt{PARAM}, in a restricted context. The same element used in different contexts or models have different UTypes; which hinders interoperability.
  \item UTypes constrain to single-table serializations. All model elements must 
  be contained within the same VOTable TABLE. 
  \item UTypes do not support annotating the VOTable content towards multiple models 
  (as TimeSeries or Catalog/Source for instance)
\end{itemize}

The UType approach could have been extended to overcome these drawbacks, but it has been decided to move forward a solution closer to the \texttt{VODML} \citep{2018ivoa.spec.0910L} concepts. 
\texttt{VODML} is a meta-model that gives a standard way to express VO models as machine-readable XML document.
In \texttt{VODML},  model leaves are no longer identified by simple strings like UTypes do, but by the role they play in a given location in the model hierarchy.
As a consequence, an annotation mechanism such as MIVOT, based on \texttt{VODML} and preserving the model hierarchy can easily provide data representations faithful to that model.

The basis of MIVOT is to insert into the VOTable, an XML block complying with the 
model structure and containing references to the actual data.
These blocks are designed in such a way that a model-aware client can build  model instances by copying that structure and by resolving the references. 
Other clients can just ignore them and rely as usual on the FIELD/PARAM and GROUP structures.


