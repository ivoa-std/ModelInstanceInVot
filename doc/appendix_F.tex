The VOMDL standard has been designed to provide machine readable descriptions of data models. 
This feature makes MIVOT working for any model element described in this language. 
For serializing data model instances, it is therefore strongly recommended to build 
the model annotations directly from VODML resources.

IVOA models are released with the following resources:
\begin{itemize}
    \item The standard specification document (pdf)
    \item The VODML specification (model-vx.y.vodml.xml)
    \item An HTML representation of the model generated from the VODML file using a dedicated style sheet.    
\end{itemize}

Both PDF and HTML documents are accessible through the standard landing page
on \url{https://ivoa.net/documents/index.html}.
VODML model description files are accessible from \url{https://ivoa.net/xml/index.html}.

This document does not specify algorithms converting VODML objects 
into MIVOT elements. However, the snippets in the following listing illustrate the translation process for a simple PhotDM class.

\begin{lstlisting}[caption={VODML representation of the PhotDM class \texttt{Access}.
This is an object type with 3 attributes, each one with a cardinality equal  to 1. 
At this stage, we do not know whether attributes are typed complex or primitive type. 
This will be completed later on by going through their own VODML definitions.},language=XML]
  <objectType>
    <vodml-id>Access</vodml-id>
    <name>Access</name>
    <description>Gathers all properties to access a resource : uri, format and size . 
    </description>
    <attribute>
      <vodml-id>Access.reference</vodml-id>
      <name>reference</name>
      <description>URI to access the resource.</description>
      <datatype>
        <vodml-ref>ivoa:anyURI</vodml-ref>
      </datatype>
      <multiplicity>
        <minOccurs>1</minOccurs>
        <maxOccurs>1</maxOccurs>
      </multiplicity>
    </attribute>
    <attribute>
      <vodml-id>Access.size</vodml-id>
      <name>size</name>
      <description>Approximate estimated size of the dataset, specified in kilobytes.</description>
      <datatype>
        <vodml-ref>ivoa:integer</vodml-ref>
      </datatype>
      <multiplicity>
        <minOccurs>1</minOccurs>
        <maxOccurs>1</maxOccurs>
      </multiplicity>
    </attribute>
    <attribute>
      <vodml-id>Access.format</vodml-id>
      <name>format</name>
      <description>Format of the accessed resource. </description>
      <datatype>
        <vodml-ref>ivoa:string</vodml-ref>
      </datatype>
      <multiplicity>
        <minOccurs>1</minOccurs>
        <maxOccurs>1</maxOccurs>
      </multiplicity>
    </attribute>
  </objectType>
\end{lstlisting}  

Basically the VODML to MIVOT translation obeys the following rules:
\begin{itemize}
    \item a VODML <objectType> or <dataType> is converted into a MIVOT \texttt{INSTANCE}. 
    \item The VODML <datatype> of the ivoa model element is taken to set the xml attribute \texttt{@dmtype} into a MIVOT \texttt{ATTRIBUTE} or \texttt{INSTANCE}. 
    \item The VODML <vodml-id> of the ivoa model element is taken to set the xml attribute \texttt{@dmrole} into a MIVOT  \texttt{COLLECTION}, \texttt{ATTRIBUTE} or \texttt{INSTANCE}. 
    \item a VODML <attribute> with a cardinality equal to 1 and a primitive type 
          is converted into a MIVOT \texttt{ATTRIBUTE}. 
    \item a VODML <attribute> with a cardinality greater than 1 is enclosed into a MIVOT \texttt{COLLECTION}. 
    \item a VODML <composition> with a cardinality greater than 1 is enclosed into a MIVOT \texttt{COLLECTION}.     
\end{itemize} 

\begin{lstlisting}[caption={MIVOT instanciation of the PhotDM class \texttt{Access}. 
VODML attributes are mapped as simple \texttt{ATTRIBUTE}s since their cardinality is equal to 
1 and they have primitive types. In this example, automatically 
generated, \texttt{ATTRIBUTE}s come with both \texttt{@ref} and \texttt{@value}. Using one, the other or both 
depends on the actual data being annotated (see \ref{ATTRIBUTE}). 
The cryptic  "@@@@@" label is a tag used by the annoter tool to indicate a place holder for column reference. It must be replaced with the actual name of the table column (reference to a \texttt{FIELD} id or name, actually) to be used to set the \texttt{ATTRIBUTE} values.
},language=XML]
      <INSTANCE dmrole="Phot:TransmissionCurve.access" dmtype="Phot:Access">
         <ATTRIBUTE dmrole="Phot:Access.reference" dmtype="ivoa:anyURI" ref="@@@@@" value=""/>
         <ATTRIBUTE dmrole="Phot:Access.size" dmtype="ivoa:integer" unit="" ref="@@@@@" value=""/>
         <ATTRIBUTE dmrole="Phot:Access.format" dmtype="ivoa:string" ref="@@@@@" value=""/>
      </INSTANCE>
\end{lstlisting}  

The conversion rules are a bit more complex when we have to tackle with inheritance or constraints.
Abstract classes should not be mapped into MIVOT as such. They must be mapped as concrete classes 
whose type depends on the mapped data (e.g. the error type associated with one measurement).

This simple example has been generated by the \texttt{snippet\_builder} module of the MIVOT validator 
(\url{https://github.com/ivoa/mivot-validator}).

