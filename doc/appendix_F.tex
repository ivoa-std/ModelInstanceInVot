The VOMDL standard has been design to offer a machine readable serialization for models. 
This feature is the basis of the MIVOT design. 
It makes it working for any model
and therefore it is strongly recommended to build annotations from VODML resources.

IVOA models are released with the following resources:
\begin{itemize}
    \item The standard specification (pdf)
    \item The VODML specification (model-vx.y.vodml.xml)
    \item An HTML representation of the model generated from the VODML file by a style sheet.    
\end{itemize}

Both PDF and HTML are accessible through the standard landing page
on \url{https://ivoa.net/documents/index.html}.
VODML files are accessible from \url{https://ivoa.net/xml/index.html}.

The 2 snippets below show how a VOMDL complex type is translated as a MIVOT instance.

\begin{lstlisting}[caption={VODML representation of the PhotDM class \texttt{Access}.
This is an object type with 3 attributes, each with a cardinality equal  to 1. 
At this stage, we do not know whether attributes are typed with a complex types 
or primitive types. This will come later by going through their own types.},language=XML]
  <objectType>
    <vodml-id>Access</vodml-id>
    <name>Access</name>
    <description>Gathers all properties to access a resource : uri, format and size . 
    </description>
    <attribute>
      <vodml-id>Access.reference</vodml-id>
      <name>reference</name>
      <description>URI to access the resource.</description>
      <datatype>
        <vodml-ref>ivoa:anyURI</vodml-ref>
      </datatype>
      <multiplicity>
        <minOccurs>1</minOccurs>
        <maxOccurs>1</maxOccurs>
      </multiplicity>
    </attribute>
    <attribute>
      <vodml-id>Access.size</vodml-id>
      <name>size</name>
      <description>Approximate estimated size of the dataset, specified in kilobytes.</description>
      <datatype>
        <vodml-ref>ivoa:integer</vodml-ref>
      </datatype>
      <multiplicity>
        <minOccurs>1</minOccurs>
        <maxOccurs>1</maxOccurs>
      </multiplicity>
    </attribute>
    <attribute>
      <vodml-id>Access.format</vodml-id>
      <name>format</name>
      <description>Format of the accessed resource. </description>
      <datatype>
        <vodml-ref>ivoa:string</vodml-ref>
      </datatype>
      <multiplicity>
        <minOccurs>1</minOccurs>
        <maxOccurs>1</maxOccurs>
      </multiplicity>
    </attribute>
  </objectType>
\end{lstlisting}  

\begin{lstlisting}[caption={MIVOT instanciation of the PhotDM class \texttt{Access}. 
VODML attributes are mapped as simple \texttt{ATTRIBUTE} since their cardinality is equal to 
1 and they have primitive types. In this example, automatically 
generated, \texttt{ATTRIBUTE} come with both \texttt{ref} and \texttt{value}. Using one, the other or both 
depends on the data available in the VOTable being annotated (see \ref{ATTRIBUTE}). 
The cryptic  "@@@@@" labels are due to a trick of the annoter tool. They must be replaced with the 
 \texttt{FIELD} names for attributes that are set with from field values.
},language=XML]
      <INSTANCE dmrole="Phot:TransmissionCurve.access" dmtype="Phot:Access">
         <ATTRIBUTE dmrole="Phot:Access.reference" dmtype="ivoa:anyURI" ref="@@@@@" value=""/>
         <ATTRIBUTE dmrole="Phot:Access.size" dmtype="ivoa:integer" unit="" ref="@@@@@" value=""/>
         <ATTRIBUTE dmrole="Phot:Access.format" dmtype="ivoa:string" ref="@@@@@" value=""/>
      </INSTANCE>
\end{lstlisting}  

This simple example has been generated by the \texttt{snippet_builder} module of the MIVOT validator 
(\url{https://github.com/ivoa/mivot-validator}).


