Services providing annotated responses must use the \texttt{REPORT}  element to tell the client whether 
the annotation process succeeded or not. 

\begin{itemize}
\item \texttt{REPORT} is not mandatory.
\item It must be the first \texttt{VODML} child if present.
\end{itemize}

\begin{lstlisting}[caption={\texttt{REPORT} example for an valid annotation (see line~\ref{REPORT_snippet} in Appendix A).},language=XML]
 <VODML xmlns:dm-mapping="http://www.ivoa.net/xml/mivot" >
    <REPORT status="OK">Mapping compiled by hand</REPORT>
	<!-- 
	   other annotations
	  -->	
</VODML>
\end{lstlisting}

\begin{lstlisting}[caption={\texttt{REPORT} example for an annotation failure.},language=XML]
<VODML	xmlns="http://www.ivoa.net/xml/mivot">
	
	<REPORT status="FAILED">
	    Missing column for the declination, positions cannot be mapped.
	</REPORT>
	<!-- 
	   No other annotation
	  -->	
</VODML>
\end{lstlisting}


\begin{table}[!htbp]
  \small
  \centering
  \begin{tabulary}{\linewidth}{|c|J|}       
    \hline 
         \textbf{Attribute} & 
         \textbf {Role}\\
    \hline
    \hline  
         @status  & 
        Status of the annotation process; must be either \texttt{OK} or \texttt{FAILED} \\
    \hline 
  \end{tabulary}
  \caption{\texttt{REPORT} attributes.} 
  \label{tbl:report-att}
\end{table}

