Services providing annotated responses must use the \texttt{REPORT}  element to tell the client whether the annotation process succeeded or not.

\begin{itemize}
\item \texttt{REPORT} is not mandatory.
\item It must be the first \texttt{VODML} child if present.
\end{itemize}



\begin{lstlisting}[caption={\texttt{REPORT} example for an annotation failure},language=XML]
<VODML	xmlns="http://www.ivoa.net/xml/merged-syntax">
	
	<REPORT status="KO">
	    The annotation process failed
	</REPORT>
	<!-- 
	   No other annotation
	  -->	
</VODML>
\end{lstlisting}

\begin{lstlisting}[caption={\texttt{REPORT} example for an valid annotation},language=XML]
<VODML	xmlns="http://www.ivoa.net/xml/merged-syntax">
	
	<REPORT status="OK">
	    The annotation process succeed
	</REPORT>

	<MODEL name="model" url="http://aaaaaa" />
	<MODEL name="model" />
	
	<GLOBALS>...</GLOBALS>
	<TEMPLATES ...>
	   ...
	 </TEMPLATES>
	
</VODML>
\end{lstlisting}


\begin{table}[!htbp]
  \small
  \centering
  \begin{tabulary}{\linewidth}{|c|J|}       
    \hline 
         \textbf{Attribute} & 
         \textbf {Role}\\
    \hline
    \hline  
         @status  & 
        Status of the annotation process; must be either \texttt{OK} or \texttt{KO} \\
    \hline 
  \end{tabulary}
  \caption{\texttt{REPORT} attributes} 
  \label{tbl:report-att}
\end{table}

