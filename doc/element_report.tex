Services providing annotated responses must 
to be able to tell the client whether the annotation process succeeded or not.
This is the purpose of the \texttt{REPORT} element.

\begin{itemize}
\item \texttt{REPORT} is not mandatory.
\item It must be the first \texttt{VODML} child if present.
\end{itemize}



\begin{lstlisting}[frame=single,caption={Example of a REPORT element},style=XML,basicstyle=\tiny]
<dm-mapping:VODML
	xmlns:dm-mapping="http://www.ivoa.net/xml/merged-syntax">
	
	<dm-mapping:REPORT status="KO">
	    The annotation process failed
	</dm-mapping:REPORT>

	<dm-mapping:MODEL name="model" url="http://aaaaaa" />
	<dm-mapping:MODEL name="model" />
	
</dm-mapping:VODML>
\end{lstlisting}

\begin{table}[!htbp]
  \small
  \centering
  \begin{tabulary}{\linewidth}{|c|J|}       
    \hline 
         \textbf{Attribute} & 
         \textbf {Role}\\
    \hline
    \hline  
         @status  & 
        Status of the annotation process.; must be either \texttt{OK} or \texttt{KO} \\
    \hline 
  \end{tabulary}
  \caption{\texttt{REPORT} attributes} 
  \label{tbl:model-att}
\end{table}

