    \texttt{COLLECTION} is a container element to be used each time we have to gather items of the same kind.  It is used in different contexts, each allowing a limited subset of elements for its content. 
     \texttt{COLLECTION} items must all be of the same type.
    \texttt{COLLECTION} can be populated either by a static list of items (\texttt{INSTANCE}, \texttt{ATTRIBUTE}, ..) or by joining to another \texttt{COLLECTION} by using the \texttt{JOIN} statement.    
    
    \begin{enumerate}
    \item{As child of INSTANCE}
      
      The \texttt{COLLECTION} serves as a container for elements with multiplicity $>$ 1.\\
      Examples of usage in this context would be:
      \begin{itemize}
        \item an array attribute
        \item a reference relation with multiplicity $>$ 1
        \item a composition relation with multiplicity $>$ 1
      \end{itemize}
      
    \item{As child of GLOBALS}
          
      The \texttt{COLLECTION} serves as a proxy for TABLE, grouping common \texttt{INSTANCE}  for selection by \texttt{PRIMARY\_KEY}, \texttt{FOREIGN\_KEY} pairs.
      
      Examples of usage in this context would be:
      \begin{itemize}
        \item a set of photometry filters, which apply to various rows of a photometric data table, based on the value of the 'band' column.
        \item a set of Dataset metadata instances, which apply to various rows of a photometric data table, based on the value of the 'id' column.
      \end{itemize}
          
    \item{As child of COLLECTION}
    
	The collection contains a matrix of  atomic values.
        
    \end{enumerate}
   
\begin{lstlisting}[caption={Example of a \texttt{COLLECTION} child of \texttt{INSTANCE}. 
It must have a role as an enclosing \texttt{INSTANCE} component.
In this example, the collection plays the role \texttt{model:Thing.elems} within the enclosing instance.
The role is the actual VODMID of the corresponding VO-DML composition.},language=XML]
<INSTANCE dmtype="model:Thing">
    <COLLECTION dmrole="model:Thing.elems">
        <ATTRIBUTE dmtype="model:Foo" value="100" />
        <ATTRIBUTE dmtype="model:Foo" value="110" />
    </COLLECTION>
</INSTANCE>
\end{lstlisting}   

\begin{lstlisting}[caption={Example of \texttt{COLLECTION} child of \texttt{GLOBALS} (see line~\ref{COLLECTION_snippet_1} in Appendix~\ref{appendix_A}). It plays no specific role as being part of the \texttt{GLOBALS} elements.},language=XML]
<GLOBALS>
    <!--
      Container for the datasets (one per timeseries)
      -->
    <COLLECTION dmid="IDDatasets" dmrole="">
        <INSTANCE dmid="IDds1" dmtype="ds:experiment.ObsDataset">
             <PRIMARY_KEY dmtype="ivoa:string" value="5813181197970338560"/>
             <ATTRIBUTE dmrole="ds:dataset.Dataset.dataProductType" 
                        dmtype="ds:dataset.DataProductType" value="TIMESERIES"/>
             <ATTRIBUTE dmrole="ds:dataset.Dataset.dataProductSubtype" 
                        dmtype="ivoa:string" value="GAIA Time Series"/>
             <ATTRIBUTE dmrole="ds:experiment.ObsDataset.calibLevel" 
                        dmtype="ivoa:integer" value="1"/>
             <REFERENCE dmrole="ds:experiment.ObsDataset.target" dmref="IDtg1"/>
        </INSTANCE>
        
        <INSTANCE dmid="IDds1" dmtype="ds:experiment.ObsDataset">
             <PRIMARY_KEY dmtype="ivoa:string" value="5813181197970338561"/>
              ...
        </INSTANCE>
    </COLLECTION>
</GLOBALS>
\end{lstlisting}   

\begin{lstlisting}[caption={Example of \texttt{COLLECTION} populated with a JOIN 
(see line~\ref{COLLECTION_snippet_2} in Appendix~\ref{appendix_A}).
The collection will be populated with the rows of another collection of the MIVOT block 
identified by \texttt{@dmid=IDtsIDdata}
and that match both \texttt{WHERE} conditions.
},language=XML]
<TEMPLATES tableref="IDPKTable">
    <INSTANCE dmid="IDTimeSeries" dmrole="" dmtype="cube:SparseCube">
        ....
        <COLLECTION dmrole="cube:SparseCube.data">
            <JOIN dmref="IDtsIDdata">
                <WHERE foreignkey="IDsrcid" primarykey="IDpksrcid" />
                <WHERE foreignkey="IDband" primarykey="IDpkband"  />
            </JOIN>
         </COLLECTION>
     </INSTANCE>
</TEMPLATES>
\end{lstlisting}   

\begin{lstlisting}[caption={Example of \texttt{COLLECTION} mapping a 2x2 matrix.},language=XML]
<TEMPLATES>
  <INSTANCE dmtype="model:matrix_22">
    <COLLECTION dmrole="model:matrix">
    	<COLLECTION>
        	<ATTRIBUTE dmtype="ivoa:real" ref="field_11"/>
        	<ATTRIBUTE dmtype="ivoa:real" ref="field_12"/>
    	</COLLECTION>
    	<COLLECTION>
        	<ATTRIBUTE dmtype="ivoa:real" ref="field_21"/>
        	<ATTRIBUTE dmtype="ivoa:real" ref="field_22"/>
    	</COLLECTION>
    </COLLECTION>
  </INSTANCE>
</TEMPLATES>
\end{lstlisting}    

\begin{table}[!htbp]
  \small
  \centering
  \begin{tabulary}{\linewidth}{|c|J|}       
    \hline 
         \textbf{Attribute} & 
         \textbf {Role}\\
    \hline
    \hline  
         \texttt{@dmid} & 
         Datamodel element id, MUST be unique within the mapping block.\\
    \hline 
         \texttt{@dmrole} & 
         Role of the \texttt{COLLECTION} in the data model. \\
    \hline 
  \end{tabulary}
  \caption{\texttt{XML attributes for COLLECTION} .} 
  \label{tbl:collection-att}
 \end{table}

\begin{table}[!htbp]
  \small
  \centering
  \begin{tabulary}{\linewidth}{|c|c|c|J|}
    \hline 
      \textbf{Context} &
      \textbf{@dmid} &
      \textbf{@dmrole} &
      \textbf{Pattern}\\
    \hline      \hline  
      1 &
      OPT & 
      MAND & 
      The element maps a collection playing a role in a modeled \texttt{INSTANCE}.  \texttt{@dmrole} MUST NOT be empty.  If present, \texttt{@dmid} MUST NOT be empty. \\
    \hline   
      2 &
      MAND & 
      NO & 
      The collection has no role. It MUST have non-empty  \texttt{@dmid} to reference for ORM selection of contained \texttt{INSTANCE}. This occurs when e.g. the \texttt{COLLECTION}  is child  of  \texttt{GLOBALS}\\
       \hline   
      3 &
      OPT& 
      NO & 
      The element maps a collection within another \texttt{COLLECTION}.  \texttt{@dmrole} MUST be empty or not set.  If present, \texttt{@dmid} MUST NOT be empty. \\
    \hline 
  \end{tabulary}
  \caption{Valid XML attribute patterns for \texttt{COLLECTION}. } 
  \label{tbl:collection-pattern}
 \end{table}

\begin{table}[!htbp]
  \small
  \centering
  \begin{tabulary}{\linewidth}{|c|c|J|}
    \hline 
      \multicolumn{3}{|l|}{\textbf{Context: Child of INSTANCE}} \\
    \hline 
      \textbf{Element} & \textbf{Occurs} & \\
    \hline
    \hline  
        \texttt{ATTRIBUTE} & 0-* & Collection of attributes.\\
    \hline    
        \texttt{REFERENCE} & 0-* & Collection of references.\\
    \hline    
        \texttt{INSTANCE} &  0-* &  Collection of instances.\\
    \hline    
        \texttt{JOIN} & 0-1 & Collection populated with a set of joined instances.\\
                 &        & No other child is supported in this case.\\
    \hline    
        \texttt{COLLECTION} & 0-* & Collection of collections.\\
    \hline    
    \hline 
      \multicolumn{3}{|l|}{\textbf{Context: Child of GLOBALS}} \\
    \hline 
      \textbf{Element} &\textbf{Occurs} &  \\
    \hline
    \hline    
        \texttt{INSTANCE} & 0-* & Collection of related instances.\\
    \hline    
        \texttt{REFERENCE} & 0-* & Collection of related references.\\
    \hline    
        \texttt{JOIN} & 0-1 & Collection populated with a set of joined instances.\\
                 &        & No other child is supported in this case.\\
    \hline  
  \end{tabulary}
     \caption{Allowed children for \texttt{COLLECTION}.} 
     \label{tbl:collection-chilren}
 \end{table}
