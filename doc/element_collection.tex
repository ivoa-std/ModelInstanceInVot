    \texttt{COLLECTION} is a container element.  It is used in different contexts, each allowing a limited subset of elements for its content. 
     \texttt{COLLECTION} items must all be of the same type.
    \texttt{COLLECTION} can be populated either by a static list of items (\texttt{INSTANCE}, \texttt{ATTRIBUTE}, ..) or by joining to another \texttt{COLLECTION} by using the \texttt{JOIN} statement.    
    
    \begin{enumerate}
    \item{As child of INSTANCE}
      
      The \texttt{COLLECTION} serves as a container for elements with multiplicity $>$ 1.\\
      Examples of usage in this context would be:
      \begin{itemize}
        \item an array attribute
        \item a reference relation with multiplicity $>$ 1
        \item a composition relation with multiplicity $>$ 1
      \end{itemize}
      
    \item{As child of GLOBALS}
          
      The \texttt{COLLECTION} serves as a proxy for TABLE, grouping common \texttt{INSTANCE}  for selection by PRIMARY \texttt{FOREIGN\_KEY}.
      
      Examples of usage in this context would be:
      \begin{itemize}
        \item a set of photometry filters, which apply to various rows of a photometric data table, based on the value of the 'band' column.
        \item a set of Dataset metadata instances, which apply to various rows of a photometric data table, based on the value of the 'id' column.
      \end{itemize}
          
    \item{As child of COLLECTION}
    
	The collection contains a matrix of  atomic values.
        
    \end{enumerate}
   
\begin{lstlisting}[caption={Example of \texttt{COLLECTION} child of \texttt{INSTANCE}.},language=XML]
<INSTANCE dmtype="model:Thing">
    <COLLECTION dmrole="model:Thing.elems">
        <ATTRIBUTE dmtype="model:Foo" value="100" />
        <ATTRIBUTE dmtype="model:Foo" value="110" />
    </COLLECTION>
</INSTANCE>
\end{lstlisting}   

\begin{lstlisting}[caption={Example of \texttt{COLLECTION} child of \texttt{GLOBALS}.},language=XML]
<GLOBALS>
    <COLLECTION dmid="_filters" >
        <INSTANCE dmtype="model:PhotometryFilter" >
            <PRIMARY_KEY dmtype="ivoa:string" value="RP"/>
            <ATTRIBUTE dmrole="model:PhotometryFilter.name" dmtype="ivoa:string"
                                    value="GAIA/GAIA2r.Grp"/>
        </INSTANCE>
        <INSTANCE dmtype="model:PhotometryFilter" >
            <PRIMARY_KEY dmtype="ivoa:string" value="BP"/>
            <ATTRIBUTE dmrole="model:PhotometryFilter.name" dmtype="ivoa:string"
                                    value="GAIA/GAIA2r.Gbp"/>
        </INSTANCE>
    </COLLECTION>
<GLOBALS>
\end{lstlisting}   

\begin{lstlisting}[caption={Example of \texttt{COLLECTION} populated with a JOIN. See Appendix..~\ref{appen_join}.},language=XML]
<GLOBALS>
    <COLLECTION dmrole="_joined_dataj">
        <JOIN tableref="_someTemplates" dmref="_extInst"/>
    </COLLECTION>
<GLOBALS>
\end{lstlisting}   

\begin{table}[!htbp]
  \small
  \centering
  \begin{tabulary}{\linewidth}{|c|J|}       
    \hline 
         \textbf{Attribute} & 
         \textbf {Role}\\
    \hline
    \hline  
         \texttt{@dmid} & 
         Datamodel element id, MUST be unique within the document.\\
    \hline 
         \texttt{@dmrole} & 
         Role of the \texttt{COLLECTION} in the data model. \\
    \hline 
  \end{tabulary}
  \caption{\texttt{XML attributes for COLLECTION} .} 
  \label{tbl:collection-att}
 \end{table}

\begin{table}[!htbp]
  \small
  \centering
  \begin{tabulary}{\linewidth}{|c|c|c|J|}
    \hline 
      \textbf{Context} &
      \textbf{@dmid} &
      \textbf{@dmrole} &
      \textbf{Pattern}\\
    \hline      \hline  
      1 &
      OPT & 
      MAND & 
      The element maps a collection playing a role in a modeled \texttt{INSTANCE}.  \texttt{@dmrole} MUST NOT be empty.  If present, \texttt{@dmid} MUST NOT be empty. \\
    \hline   
      2 &
      MAND & 
      NO & 
      The collection has no role. It MUST have non-empty  \texttt{@dmid} to reference for ORM selection of contained \texttt{INSTANCE}. This occurs when e.g. the \texttt{COLLECTION}  is child  of  \texttt{GLOBALS}\\
       \hline   
      3 &
      OPT& 
      NO & 
      The element maps a collection within another \texttt{COLLECTION}.  \texttt{@dmrole} MUST not set or empty.  If present, \texttt{@dmid} MUST NOT be empty. \\
    \hline 
  \end{tabulary}
  \caption{Valid XML attribute patterns for \texttt{COLLECTION}. } 
  \label{tbl:collection-pattern}
 \end{table}

\begin{table}[!htbp]
  \small
  \centering
  \begin{tabulary}{\linewidth}{|c|c|J|}
    \hline 
      \multicolumn{3}{|l|}{\textbf{Context: Child of INSTANCE}} \\
    \hline 
      \textbf{Element} & \textbf{Occurs} & \\
    \hline
    \hline  
        \texttt{ATTRIBUTE} & 0-* & Collection of attributes.\\
    \hline    
        \texttt{REFERENCE} & 0-* & Collection of references.\\
    \hline    
        \texttt{INSTANCE} &  0-* &  Collection of instances.\\
    \hline    
        \texttt{JOIN} & 0-1 & Collection populated with a set of joined instances.\\
                 &        & No other child is supported in this case.\\
    \hline    
        \texttt{COLLECTION} & 0-* & Collection of collections.\\
    \hline    
    \hline 
      \multicolumn{3}{|l|}{\textbf{Context: Child of GLOBALS}} \\
    \hline 
      \textbf{Element} &\textbf{Occurs} &  \\
    \hline
    \hline    
        \texttt{INSTANCE} & 0-* & Collection of related instances.\\
    \hline    
        \texttt{REFERENCE} & 0-* & Collection of related references.\\
    \hline  
  \end{tabulary}
     \caption{Allowed children for \texttt{COLLECTION}.} 
     \label{tbl:collection-chilren}
 \end{table}
