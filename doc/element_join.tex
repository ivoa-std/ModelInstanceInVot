The \texttt{JOIN} element allows to populate a \texttt{COLLECTION} with \texttt{INSTANCE}  from another collection, namely the foreign collection.
The foreign collection can either be a static element  \texttt{GLOBALS} \texttt{COLLECTION}  or a collection of \texttt{INSTANCE}  resulting from the iteration over a \texttt{TEMPLATES} 

The \texttt{JOIN} element must contain attributes identifying the foreign collection   \texttt{@sourceref}  and/or \texttt{@dmref} . 
It can have \texttt{WHERE} children stating the join condition and it must be the unique child of the host \texttt{COLLECTION} 
 \texttt{JOIN} may have 2 uses:

\begin{itemize}

    \item Join with \texttt{TEMPLATES} data:
       \begin{itemize}
         \item must have either a \texttt{@sourceref} attribute identifying the foreign \texttt{TEMPLATES} or a \texttt{@dmref} attribute identifying the foreign mapped \texttt{INSTANCE} or both.
             \begin{itemize}
               \item if only \texttt{@sourceref} is given, the host collection is populated with the \texttt{INSTANCE}  mapping the foreign table rows. 
                        This pattern is only valid when the foreign TEMPLATEs has  one unique instance child.
               \item if only \texttt{@dmref}  the host collection is populated with the \texttt{INSTANCE}  having the matching \texttt{@dmid}  In this case, 
                        the client is in charge of locating the \texttt{TEMPLATES} containing that \texttt{INSTANCE} 
               \item if both \texttt{@sourceref} and \texttt{@dmref} are given, it is assumed that the \texttt{INSTANCE} with \texttt{@dmid} matching \texttt{@dmref} is hosted by the TEMPLATEs matching \texttt{@sourceref} 
                        This over-statement may help the parser but it can carry inconsistencies. An error must be risen in this case.
             \end{itemize}
    \end{itemize}
  
   \item Join with \texttt{GLOBALS} \texttt{COLLECTION} data:
       \begin{itemize}
         \item the foreign \texttt{COLLECTION} must be direct child of \texttt{GLOBALS} 
         \item \texttt{JOIN} must have one \texttt{@sourceref} attribute identifying the foreign \texttt{COLLECTION} 
         \item the host \texttt{COLLECTION} is populated with the foreign \texttt{COLLECTION} items eventually filtered by the \texttt{WHERE} condition.
         \item the foreign \texttt{COLLECTION} must only have \texttt{INSTANCE}  (eventually \texttt{REFERENCE}) as children.
         \item the foreign \texttt{COLLECTION} items are supposed to have all the same \texttt{@dmtype}  this rule is however out of the scope of the present standard.
         \item If there are \texttt{WHERE} conditions, the foreign \texttt{COLLECTION} items must have \texttt{PRIMARY\_KEY} 
  \end{itemize}
\end{itemize}

\begin{lstlisting}[label={lst:join},caption={\texttt{JOIN example whit 2 join conditions} },language=XML]
<JOIN sourceref="Results" dmref="_ts_data">
  <WHERE foreignkey="_srcid" primarykey="_pksrcid" />
  <WHERE foreignkey="_band" primarykey="_pkband" />
</JOIN>
\end{lstlisting}


\begin{table}[!htbp]
\small
\centering
\begin{tabulary}{\linewidth}{|c|J|}       
       \hline 
            \textbf{Attribute} & 
            \textbf {Role}\\
       \hline         \hline  
             \texttt{@sourceref} & 
            Reference of the \texttt{TEMPLATES} or \texttt{COLLECTION} to be joined with. \\
        \hline 
            \texttt{@dmref} & 
            Reference of the foreign \texttt{INSTANCE}s that will populate he host \texttt{COLLECTION}  \\
        \hline 
     \end{tabulary}
     \caption{\texttt{JOIN} attributes} 
     \label{tbl:join-att}
 \end{table}

\begin{table}[!htbp]
\small
\centering
\begin{tabulary}{\linewidth}{|c|c|J|}
    \hline 
        \textbf{@sourceref } &
        \textbf{@dmref} &
        \textbf{Pattern}\\
    \hline      \hline  
        MAND &           
        OPT &           
        Join against the \texttt{TEMPLATES} or \texttt{COLLECTION} identified by \texttt{@sourceref}  \\  
    \hline   
        OPT &           
        MAND &      
        Host \texttt{COLLECTION} populated with \texttt{INSTANCE}  identified by \texttt{@dmref} \\
   \hline 
\end{tabulary}
     \caption{Valid attribute patterns for  \texttt{JOIN}}
     \label{tbl:join-pattern}
\end{table}


\begin{table}[!htbp]
\small
\centering
\begin{tabulary}{\linewidth}{|c |c |c|J|}
    \hline 
        \textbf{Element} &
        \textbf{Position} &
        \textbf{Occurs} &
        \\
    \hline      \hline  
        \texttt{WHERE}  &        
        1 &           
        0-* &
         Join condition\\
    \hline 
\end{tabulary}
     \caption{Allowed children for \texttt{JOIN}} 
     \label{tbl:join-chidlren}
 \end{table}
