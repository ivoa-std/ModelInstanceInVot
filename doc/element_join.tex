The JOIN element allows to populate a COLLECTION with INSTANCEs from another collection, namely the foreign collection.
The foreign collection can either be a static element (GLOBALS/COLLECTION) or a collection of INSTANCES resulting from the iteration over a TEMPLATES.

The JOIN element must contain attributes identifying the foreign collection  (@sourceref  and/or @dmref). 
It can have WHERE children stating the join condition and it must be the unique child of the host COLLECTION.

JOIN may have 2 uses:

\begin{itemize}

    \item Join with TEMPLATES data:
       \begin{itemize}
         \item must have either a @sourceref attribute identifying the foreign TEMPLATES or a @dmref attribute identifying the foreign mapped INSTANCE or both.
             \begin{itemize}
               \item if only @sourceref is given, the host collection is populated with the INSTANCEs mapping the foreign table rows. 
                        This pattern is only valid when the foreign TEMPLATEs has  one unique instance child.
               \item if only @dmref, the host collection is populated with the INSTANCEs having the matching @dmid. In this case, 
                        the client is in charge of locating the TEMPLATES containing that INSTANCE.
               \item if both @sourceref and @dmref are given, it is assumed that the INSTANCE with @dmid matching @dmref is hosted by the TEMPLATEs matching @sourceref.
                        This over-statement may help the parser but it can carry inconsistencies. An error must be risen in this case.
             \end{itemize}
    \end{itemize}
  
   \item Join with GLOBALS/COLLECTION data:
       \begin{itemize}
         \item the foreign COLLECTION must be direct child of GLOBALS.
         \item JOIN must have one @sourceref attribute identifying the foreign COLLECTION.
         \item the host COLLECTION is populated with the foreign COLLECTION items eventually filtered by the WHERE condition.
         \item the foreign COLLECTION must only have INSTANCEs (eventually REFERENCES) as children.
         \item the foreign COLLECTION items are supposed to have all the same @dmtype, this rule is however out of the scope of the present standard.
         \item If there are WHERE conditions, the foreign COLLECTION items must have PRIMARY\_KEYs
  \end{itemize}
\end{itemize}

\begin{lstlisting}[frame=single,label={lst:join},caption={\texttt{JOIN example whit 2 join conditions} },style=XML,basicstyle=\tiny]
<dm-mapping:JOIN sourceref="Results" dmref="_ts_data">
  <dm-mapping:WHERE foreignkey="_srcid" primarykey="_pksrcid" />
  <dm-mapping:WHERE foreignkey="_band" primarykey="_pkband" />
</dm-mapping:JOIN>
\end{lstlisting}


\begin{table}[!htbp]
\small
\centering
\begin{tabulary}{\linewidth}{|c|J|}       
       \hline 
            \textbf{Attribute} & 
            \textbf {Role}\\
       \hline         \hline  
             @sourceref & 
            Reference of the TEMPLATES or COLLECTION to be joined with. \\
        \hline 
            @dmref & 
            Reference of the foreign \texttt{INSTANCE}s that will populate he host COLLECTION. \\
        \hline 
     \end{tabulary}
     \caption{\texttt{JOIN} attributes} 
     \label{tbl:join-att}
 \end{table}

\begin{table}[!htbp]
\small
\centering
\begin{tabulary}{\linewidth}{|c|c|J|}
    \hline 
        \textbf{@sourceref } &
        \textbf{@dmref} &
        \textbf{Pattern}\\
    \hline      \hline  
        MAND &           
        OPT &           
        Join against the TEMPLATES or COLLECTION identified by @sourceref  \\  
    \hline   
        OPT &           
        MAND &      
        Host COLLECTION populated with \texttt{INSTANCE}  identified by @dmref \\
   \hline 
\end{tabulary}
     \caption{Valid attribute patterns for  \texttt{JOIN}}
     \label{tbl:join-pattern}
\end{table}


\begin{table}[!htbp]
\small
\centering
\begin{tabulary}{\linewidth}{|c |c |c|J|}
    \hline 
        \textbf{Element} &
        \textbf{Position} &
        \textbf{Occurs} &
        \\
    \hline      \hline  
        \texttt{WHERE}  &        
        1 &           
        0-* &
         Join condition\\
    \hline 
\end{tabulary}
     \caption{Allowed children for \texttt{JOIN}} 
     \label{tbl:join-chidlren}
 \end{table}
