The JOIN element allows to populate a COLLECTION with INSTANCEs from another collection, namely the foreign collection.
The foreign collection can either be a static element (GLOBALS/COLLECTION) or a collection of INSTANCES resulting from the iteration over a TEMPLATES.

The JOIN element must contain attributes identifying the foreign collection  (@tableref and/or @dmref). It can have WHERE children stating the join condition.

It must be child of a COLLECTION that has not other children element that INSTANCE. A COLLECTION cannot host more that 1 JOIN.

JOIN may have 2 uses:

\begin{itemize}

  \item Join with TEMPLATES data:
       \begin{itemize}
         \item must have a @tableref attribute identifying the foreign TEMPLATES
         \item may have a @dmref attribute the identify an INSTANCE within the foreign TEMPLATES.
       \end{itemize}
       The mapping of the enclosing COLLECTION items can be either implicit or explicit:
       \begin{itemize}
         \item If it has no other children, it must be populated with the whole table rows.
         \item Otherwise the item content is given by the enclosed INSTANCE. 
                 The reference (@ref) contained in those INSTANCEs must refer to te field of the foreign table.
       \end{itemize}
       
  \item Join with COLLECTION data:
       \begin{itemize}
         \item must have no @tableref attribute, referencing a table in this case is irrelevant.
         \item must have a @dmref attribute the identify the COLLECTION to be joined with. This COLLECTION must be GLOBALS child.
       \end{itemize}
       The mapping of the enclosing COLLECTION items can be either implicit or explicit:
       \begin{itemize}
         \item If it has no other children, it must be populated with the foreign collection items.
         \item Otherwise the item content is given by the enclosed INSTANCEs. The reference (@ref) contained in those INSTANCEs must refer to te field of the foreign table, we must not have such @ref if the foreign collection is static
       \end{itemize}
           
\end{itemize}

\begin{lstlisting}[frame=single,caption={\texttt{JOIN} },style=XML,basicstyle=\tiny]

\end{lstlisting}


\begin{table}[!htbp]
\small
\centering
\begin{tabulary}{\linewidth}{|c|J|}       
       \hline 
            \textbf{Attribute} & 
            \textbf {Role}\\
       \hline         \hline  
             @tableref& 
            Reference of the table to be joined with. \\
        \hline 
            @dmref & 
            Reference of the \texttt{COLLECTION} (in \texttt{GLOBALS} to be joined with. \\
        \hline 
     \end{tabulary}
     \caption{\texttt{JOIN} attributes} 
     \label{tbl:join-att}
 \end{table}

\begin{table}[!htbp]
\small
\centering
\begin{tabulary}{\linewidth}{|c|c|J|}
    \hline 
        \textbf{@tableref} &
        \textbf{@dmref} &
        \textbf{Pattern}\\
    \hline      \hline  
        MAND &           
        NO &           
        The join is done against the table identified by @tableref \\
    \hline   
        NO &           
        MAND &           
        The join is done against the \texttt{COLLECTION} identified by @rmref \\
   \hline 
\end{tabulary}
     \caption{Valid attribute patterns for  \texttt{JOIN}}
     \label{tbl:join-pattern}
\end{table}


\begin{table}[!htbp]
\small
\centering
\begin{tabulary}{\linewidth}{|c |c |c|J|}
    \hline 
        \textbf{Element} &
        \textbf{Position} &
        \textbf{Cardinality} &
        \\
    \hline      \hline  
        \texttt{WHERE}  &        
        1 &           
        0-* &
         Join condition\\
    \hline 
\end{tabulary}
     \caption{Allowed children for \texttt{JOIN}} 
     \label{tbl:join-chilren}
 \end{table}