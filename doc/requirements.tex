\begin {itemize}
  \item Shy annotation: model annotations come in a workflow that works very well for years, this is why the first requirement is to no break any existing stack-holders.
  \begin {itemize}
    \item The annotation must not alter the VOTable content.
    \item The annotation block must be located in a way that it can easily be skipped by model-unaware clients.
    \item The vocabulary in the annotation name-space must not overlap the VOTable elements (names or attributes)    
  \end {itemize}
  
  \item Schema and validation:
  \begin {itemize}
    \item The annotation schema must be independent from the VOTable schema
    \item The evolution of the annotation schema must not impact the VOTable schema
    \item The annotation syntax must be validated with standard tools usable in any langage
  \end {itemize}
  
  \item Model agnostic:
  \begin {itemize}
    \item The annotation syntax must be able to map data on any VODML compliant model
    \item The annotation syntax must allow client to use their own strategy to consume mapped data:
      \begin {itemize}
        \item just ignore it
        \item just pick some elements of interest 
        \item just pick model meta-data and process data stream as usual
        \item pick whole model instances
      \end {itemize}
  \end {itemize}
  
\end {itemize}