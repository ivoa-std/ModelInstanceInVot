The meta-data currently supported by the VOTable gives a very good description of the physical content of the table fields. 
There is however no satisfactory way to specify the exact role played by those quantities in a given context.
There is also no way to describe how different quantities or different data tables relate to each others.
Utypes have been proposes to fill these gaps, but the lack of standard method to build them caused this approach to be abandoned.
Actually, both role specifications and quantity associations are part of the modelling effort and what we need is a language acting as a bridge between model leaves and data columns.
This would allow to either directly read data as we do now, or to interpret it through a model view to get a better understanding of their complexity.
This dual approach does not mean that the client has to choose between a regular data processing and a full object approach.
There are intermediate levels of use of model annotations that correspond to concrete use cases and to which this standard provides an answer:

\begin{itemize}
  \item Coordinate or calibration Systems: The recent modelling efforts (Coord and PhotDM) provide a very accurate description of coordinate frames or Photometric system that need to be reproduced in VOTables.
  \begin{itemize}
    \item to make the best of the dataset  clients may need to get an exact representation of these elements as they are modelled.
    \item data set can contain the multiple quantities expressed in the same coordinate system (corrected position vs raw position) or 
             quantities of the same nature but expressed in different systems (coord IRCS vs Gal). VOTable currently do not support the latest 2 pattern
  \end{itemize} 
  
  \item Some quantities can be made with multiple components that must be gather by the client to be properly used.
  \begin{itemize}
    \item value-error associations. 
    \item errors values split in several columns (covariance matrices). 
    \item quantities with quality flags
    \item position with proper motion to compute error ellipses or position at a given epoch
  \end{itemize} 

  \item The above cases relate to the processing of individual datasets; the model annotations become even more important for cases where a good level of interoperability is required. 
           This is achieved when a given quantity is represented by the same data structure whatever its origin. This is the purpose of the Meas model that proposes classes for 
           any physical quantities and that can be rendered by the mapping syntax. Meas classes are not meant to be used as standalone elements but as parts of some host model (Cube Mango);
           however the clients keep free to either process those host models as a whole or to chase individual components.
    \begin{itemize}
      \item cross-matching is made easier if the processor has a common representation of the positions with all their components.
      \item Sed can be built with from fluxes extracted from different datasets but with the same representation.
   \end{itemize}          

  \item I some cases, we need to be able to extract complete model instances from VOTables.
    \begin{itemize}
      \item Extracting a TimeSeries instance to feed a software built upon classes generated from that model.
      \item building model instances that can be serialised in another format e.g. json to ne share in another context than the VOTable processing e.g. SAMP
   \end{itemize}          
    
\end{itemize} 

