The MIVOT syntax is controlled by an XSD schema.
This is part of this standard and available at \url{http://ivoa.net/XML/MIVOT/mivot-v1.0.xsd}.
The XML schema conforms to XSD 1.1 \citep{std:xsd1.1} which enforces the syntax rules. 
The following features are validated:

\begin{itemize} 
  \item Element names 
  \item Element attributes
  \item Element sequences 
  \item Element ordering in specific sequences
\end{itemize}

In addition to this basic check, XSD 1.1 asserts which patterns are allowed or forbidden:

\begin{itemize} 
  \item Which elements are mandatory, optional  or forbidden in the local context (parent and children elements).
  \item Which attributes are mandatory, optional  or forbidden in the context of the host element.
  \item Attribute value requirements according to the context of the host element.
\end{itemize}
 
The scope of this schema-based validation is limited to the syntax however. 
The clients are still responsible for checking whether or not the attribute values are correct and for managing any semantic inconsistencies:

\begin{itemize} 
  \item Are references resolvable ?
  \item Do units conform to any requirement of the mapped models ?
  \item Is the mapping structure faithful to the referenced models?
\end{itemize}


