The MIVOT standard comes with an XML schema conform to XSD 1.1 \citep{std:xsd1.1} that enforces the syntax rules. 
The following features are validated:

\begin{itemize} 
  \item Element names 
  \item Element attributes
  \item Element sequences 
  \item Element ordering in specific sequences
\end{itemize}

In addition to this basic check, XSD 1.1 allows for the refine the definition of the elements or attributes  patterns that are allowed or not:

\begin{itemize} 
  \item Which elements are mandatory, optional  or forbidden in the local context  (father and children elements).
  \item Which attributes are mandatory, optional  or forbidden in to the context of the host element.
  \item Attribute values required according to the context of the host element.
\end{itemize}
 
The scope of this schema-based validation is limited to the syntax however. 
The clients are still responsible for checking whether or not the attribute values are correct and for managing any semantic inconsistencies:

\begin{itemize} 
  \item References are resolvable
  \item Units are conform with the VO
  \item The mapping structure is faith to the referenced models
\end{itemize}


