A VOTable can provide serializations for an arbitrary number of data models.
In order to declare which models are represented in the file, data
providers must declare them through the \texttt{MODEL} elements.
All models that define vodml-ids used in the annotation must be declared.

\begin{lstlisting}[caption={Example \texttt{MODEL} mapping block. (see line~\ref{MODEL_snippet} in Appendix~\ref{appendix_A}) },language=XML]
<MODEL name="ivoa"   
       url="https://www.ivoa.net/xml/VODML/IVOA-v1.vo-dml.xml" />
<MODEL name="mango"
       url="https://github.com/ivoa-std/MANGO/blob/master/vo-dml/mango.vo-dml.xml" />
<MODEL name="cube"
       url="https://github.com/ivoa-std/Cube/vo-dml/Cube-1.0.vo-dml.xml" />
<MODEL name="ds"
       url="https://github.com/ivoa-std/DatasetMetadata/vo-dml/DatasetMetadata-1.0.vo-dml.xml" />
<MODEL name="coords" 
       url="https://www.ivoa.net/xml/VODML/Coords-v1.vo-dml.xml" />
<MODEL name="meas"   
       url="https://www.ivoa.net/xml/VODML/Meas-v1.vo-dml.xml" />
\end{lstlisting}

\begin{table}[!htbp]
  \small
  \centering
  \begin{tabulary}{\linewidth}{|c|J|}       
    \hline 
         \textbf{Attribute} & 
         \textbf {Role}\\
    \hline
    \hline  
         \texttt{@name}  & 
         Name of the mapped model as declared in the VO-DML/XML model serialization.  
         This attribute MUST not be empty. 
         Its value is used to prefix the values of both \texttt{@dmrole} and \texttt{@dmtype} attributes of elements from that model.  \\
    \hline 
         \texttt{@url} & 
         URL to the VO-DML/XML serialization of the model. If present, this attribute MUST not be empty.\\
    \hline 
  \end{tabulary}
  \caption{\texttt{MODEL} attributes.} 
  \label{tbl:model-att}
\end{table}


\begin{table}[!htbp]
  \small
  \centering
  \begin{tabulary}{\linewidth}{|c |c |J|}
    \hline 
        \textbf{@name} &
        \textbf{@url} &
        \textbf{Pattern}\\
    \hline      \hline  
        MAND &           
        OPT &           
        Unique attribute pattern supported by \texttt{MODEL}\\
    \hline 
  \end{tabulary}
  \caption{Valid attribute patterns for  \texttt{MODEL}.} 
  \label{tbl:model-pattern}
\end{table}
