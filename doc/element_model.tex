A VOTable can provide serializations for an arbitrary number of data model
types. In order to declare which models are represented in the file, data
providers must declare them through the MODEL elements.
Only models that are used in the file must be declared. A model is
used if at least one element in the mapping block refer to it. In other terms, only models that define vodml-ids used in the
annotation must be declared.
\begin{table}[!htbp]
\small
\centering
\begin{tabulary}{\linewidth}{|c|J|}       
       \hline 
            \textbf{Attribute} & 
            \textbf {Role}\\
       \hline         \hline  
            @name  & 
            Name of the mapped model (informal).  This attribute cannot be left empty  \\
       \hline 
            @url & 
            Url of the vo-dml serialization of the model. This attribute cannot be left empty if present.\\
       \hline 
     \end{tabulary}
     \caption{\texttt{MODEL} attributes} 
     \label{tbl:model-att}
 \end{table}


\begin{table}[!htbp]
\small
\centering
\begin{tabulary}{\linewidth}{|c |c |J|}
    \hline 
        \textbf{@name} &
        \textbf{@url} &
        \textbf{Pattern}\\
    \hline      \hline  
        MAND &           
        OPT &           
        Unique attribute pattern supported by \texttt{MODEL}\\
   \hline 
\end{tabulary}
     \caption{Valid attribute patterns for  \texttt{MODEL}} 
     \label{tbl:model-pattern}
 \end{table}