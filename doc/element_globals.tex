Some annotations may map the Resource contents to instances or collections of data model
types that are global in the mapping scope, possibly because such instances are refer-
enced by other intances that annotate specific tables. More generally, some
annotations will define instances that are completely defined in terms of
constant value, i.e. they are not represented in tabular form. Rather, they
are completely and directly represented by an XML element.
Such instances should be included in the GLOBALS element.
GLOBALS must only contain direct representations of instances, i.e.
INSTANCE elements that do not refer to any FIELD directly. 
This rule is not enforces via the XSD schema 
Also, GLOBALS should not contain any INSTANCEs with REFERENCEs to
indirect INSTANCEs.

\begin{table}[!htbp]
\small
\centering
\begin{tabulary}{\linewidth}{|c |c |c|}
    \hline 
        \textbf{Element} &
        \textbf{Position} &
        \textbf{Cardinality}\\
    \hline      \hline  
        \texttt{INSTANCE}          
        Any &           
        0-*\\
    \hline    
        \texttt{COLLECTION} &           
        Any &           
        0-*\\
    \hline 
\end{tabulary}
     \caption{Allowed children for \texttt{GLOBALS}} 
     \label{tbl:globals-chilren}
 \end{table}