The \texttt{INSTANCE} element defines a complex ObjectType or DataType.
It may be a child of several other elements, and the requirements on
the content, especially  \texttt{@dmid} and  \texttt{@dmrole}, may differ depending on
the usage:


\begin{enumerate}
\item Child of \texttt{GLOBALS}

   In this case the \texttt{INSTANCE} is a single stand-alone instance which
   may or may not be referenced by other \texttt{INSTANCE}. This pattern is typically used for 
   shared object e.g. space frames, or for head object of science product models e.g. Time Series.
   Here then \texttt{INSTANCE}
  \begin{itemize}
     \item May have  \texttt{@dmid} attribute, as possible target of \texttt{REFERENCE} ref
     \item must have no  \texttt{@dmrole} attribute or an empty one
  \end{itemize}  
     
\item Child of \texttt{TEMPLATES}

  In this case, the \texttt{INSTANCE} is a template for instances which
  are generated once per row of the associated table, which:
  \begin{itemize}
     \item may have  \texttt{@dmid} attribute, as target of \texttt{JOIN} \texttt{@dmref}
     \item must have no  \texttt{@dmrole} attribute or an empty one
  \end{itemize}  

\item Child of \texttt{COLLECTION}

  There are 2 uses for this pattern.  
  \begin{itemize}
     \item each member \texttt{INSTANCE} is a target for selection using
           the \texttt{PRIMARY/FOREIGN\_KEY} elements. This pattern is only 
           allowed within the \texttt{GLOBALS} environment. In this case it:             
           \begin{itemize}
             \item must contain at least one \texttt{PRIMARY\_KEY} sub-element
             \item may have a  \texttt{@dmid} attribute, as possible target of \texttt{REFERENCE}              
             \item must have no  \texttt{@dmrole} attribute or an empty one. \texttt{VODML} does set roles to particular collection items
           \end{itemize}

     \item each member \texttt{INSTANCE} is a cell of an element with multiplicity > 1.
          Each one :             
           \begin{itemize}
             \item must have no  \texttt{@dmrole} attribute or an empty one
           \end{itemize}
  \end{itemize}  
    
\item Child of \texttt{INSTANCE}

     In this case, each \texttt{INSTANCE} represents 
     a complex ObjectType or DataType playing a role in the parent \texttt{INSTANCE}.
     Then it     
     \begin{itemize}
        \item may have  \texttt{@dmid} attribute
        \item must have non-empty  \texttt{@dmrole} attribute
     \end{itemize}
           
\item any \texttt{INSTANCE}

   \begin{itemize}
   	 \item must have non-empty  \texttt{@dmtype} attribute
	 \item if  \texttt{@dmid} attribute is present, it must not be empty    
    \end{itemize}
\end{enumerate}  
    
   
\begin{lstlisting}[caption={Example of an \texttt{INSTANCE} child of \texttt{INSTANCE} (see line~\ref{INSTANCE_snippet} in Appendix A). It must have a role as a component of the enclosing \texttt{INSTANCE}.},language=XML]
 <INSTANCE dmid="IDtimesys" dmrole="" dmtype="coords:TimeSys">
    <PRIMARY_KEY dmtype="ivoa:string" value="TCB"/>
    <INSTANCE dmrole="coords:PhysicalCoordSys.frame" 
              dmtype="coords:TimeFrame">
        <ATTRIBUTE dmrole="coords:TimeFrame.timescale" 
                   dmtype="ivoa:string" value="TCB" />
        <INSTANCE dmrole="coords:TimeFrame.refPosition" 
                  dmtype="coords:StdRefLocation">
            <ATTRIBUTE dmrole="coords:StdRefLocation.position" 
                       dmtype="ivoa:string" value="BARYCENTER"/>
        </INSTANCE>
    </INSTANCE>
</INSTANCE>
\end{lstlisting}   
   
The various XML attributes for an \texttt{INSTANCE} element, will have constraints depending on their usage.
  
\begin{table}[!htbp]
\small
\centering
\begin{tabulary}{\linewidth}{|c|J|}       
       \hline 
            \textbf{Attribute} & 
            \textbf {Role}\\
       \hline         \hline  
            \texttt{@dmid} & 
            Element  \texttt{@dmid}  MUST be unique within the mapping block  \\
        \hline 
            \texttt{@dmrole} & 
            \texttt{INSTANCE} role played in the DM \\
        \hline 
            \texttt{@dmtype} & 
            Class name in a model context\\
        \hline 
     \end{tabulary}
     \caption{\texttt{INSTANCE} XML attributes.} 
     \label{tbl:instance-att}
 \end{table}   
 


 
\begin{table}[!htbp]
\small
\centering
\begin{tabulary}{\linewidth}{|c |c |c|J|}
    \hline 
        \textbf{Element} &
        \textbf{Position} &
        \textbf{Occurs} &
        \\
    \hline      \hline  
        \texttt{PRIMARY\_KEY}  &        
        First &           
        0-* &
        Primary key to be used to in a \texttt{JOIN} context.\\
    \hline    
        \texttt{REFERENCE}  &        
        Any &           
        0-* &
         Object attribute as a reference to either another \texttt{INSTANCE} or a \texttt{COLLECTION}. \\
    \hline    
        \texttt{INSTANCE} &           
        Any &           
        0-* &
         Object attribute as a class instance. \\
    \hline    
        \texttt{ATTRIBUTE} &           
        Any &           
        0-* &
       Object attribute as a simple attribute. \\
    \hline    
        \texttt{COLLECTION} &           
        Any &           
        0-* &
         Object attribute  as a collection.\\
    \hline 
\end{tabulary}
     \caption{Allowed children for \texttt{INSTANCE}. The Position column indicates the required rank of the child element.} 
     \label{tbl:instance-chilren}
\end{table}
 
\begin{table}[!htbp]
\small
\centering
\begin{tabulary}{\linewidth}{|c|c|c|J|}
    \hline 
        \textbf{@dmid} &
        \textbf{@dmrole} &
        \textbf{@dmtype} &
        \textbf{Pattern}\\
    \hline      \hline  
        OPT &           
        NO or EMPTY&           
        MAND &           
        MUST be applied when the  \texttt{INSTANCE} is child of \texttt{GLOBALS} or \texttt{TEMPLATES}. The element has no role because it is not embedded in a model component. It must be referable by a \texttt{REFERENCE}.  \\
    \hline   
        OPT &           
        MAND &           
        MAND &           
        MUST be applied in any other location. It may be referable by a \texttt{REFERENCE} . \\
   \hline 
\end{tabulary}
     \caption{Valid attribute patterns for  \texttt{INSTANCE}.} 
     \label{tbl:instance-pattern}
 \end{table}       
\newpage

 
