
\textbf{Mark proposal (as interpreted by LM}

   The INSTANCE element defines a complex ObjectType or DataType.
   
\begin{lstlisting}[frame=single,caption={Example of INSTANCE child of GLOBALS},style=XML,basicstyle=\tiny]
<dm-mapping:INSTANCE ID="SpaceFrame_ICRS" dmtype="coords:SpaceFrame">
	<dm-mapping:INSTANCE dmrole="coords:SpaceFrame.refPosition"
                                 dmtype="coords:StdRefLocation">
		<dm-mapping:ATTRIBUTE dmrole="coords:StdRefLocation.position" 
		                          dmtype="ivoa:string"  value="NoSet" />
	</dm-mapping:INSTANCE>
	<dm-mapping:ATTRIBUTE dmrole="coords:SpaceFrame.spaceRefFrame" 
	                          dmtype="ivoa:string" value="ICRS" />
	<dm-mapping:ATTRIBUTE dmrole="coords:SpaceFrame.equinox" 
	                          dmtype="coords:Epoch"	value="2015" />
</dm-mapping:INSTANCE>
\end{lstlisting}   
   

\begin{table}[!htbp]
\small
\centering
\begin{tabulary}{\linewidth}{|c|J|}       
       \hline 
            \textbf{Attribute} & 
            \textbf {Role}\\
       \hline         \hline  
            @ID & 
            Element ID, MUST be unique within the mapping block  \\
        \hline 
            @dmrole & 
            INSTANCE role in the DM \\
        \hline 
            @dmtype & 
            Class name \\
        \hline 
     \end{tabulary}
     \caption{\texttt{INSTANCE} attributes} 
     \label{tbl:instance-att}
 \end{table}   
    It may be a child of several other elements, and the requirements on
    the content (especially ID and dmrole), may differ depending on
    the usage:
    
\begin{itemize}
\item Child of GLOBALS:
   In this case the INSTANCE is a single stand-alone instance which
   may or may not be referenced by other INSTANCEs.
  \begin{itemize}
     \item must have ID, as possible target of REFERENCE.ref
     \item must have no or empty \texttt{dmrole}
  \end{itemize}  
  
\item Child of TEMPLATES:
  In this case, the INSTANCE is a template for instances which
  are generated once per row of the associated table.  
  \begin{itemize}
     \item may have ID, as target of JOIN.dmref
     \item must have no or empty dmrole \texttt{dmrole}
  \end{itemize}  
  
\item Child of COLLECTION:
  There are 2 uses for this pattern.  
  \begin{itemize}
     \item each member INSTANCE is a target for selection using
           the PRIMARY/FOREIGN\_KEY elements. This pattern is only 
           allowed within the GLOBALS environment. In this case:             
           \begin{itemize}
             \item must contain at least one PRIMARY\_KEY sub-element
             \item must have ID, as possible target of REFERENCE.ref
             \item must have no or empty dmrole
           \end{itemize}

     \item Elements INSTANCE are collection cells with multiplicity > 1
          Each one has:             
           \begin{itemize}
             \item must have ID, as possible target of REFERENCE.ref. 
                   this pattern is             
                   only allowed if within the GLOBALS environment
             \item must have no or empty dmrole
           \end{itemize}
    
     \item Child of INSTANCE: In this case, each INSTANCE represents 
           a complex ObjectType or DataType playing a role in the parent
           INSTANCE.     
           \begin{itemize}
             \item must not have ID (may not be referenced) ??
             \item must have non-empty dmrole
           \end{itemize}
           
     \item any INSTANCE:     
           \begin{itemize}
             \item if ID is present, it must not be empty
             \item must have non-empty dmtype
           \end{itemize}
    \end{itemize}  
  
\end{itemize}  

 
\begin{table}[!htbp]
\small
\centering
\begin{tabulary}{\linewidth}{|c |c |c|J|}
    \hline 
        \textbf{Element} &
        \textbf{Position} &
        \textbf{Cardinality} &
        \\
    \hline      \hline  
        \texttt{PRIMARY\_KEY}  &        
        First &           
        0-* &
        Primary key to be used to in a JOIN context.\\
    \hline    
        \texttt{REFERENCE}  &        
        Any &           
        0-* &
         Object attribute as a reference to either another INSTANCE or a COLLECTION.\\
    \hline    
        \texttt{INSTANCE} &           
        Any &           
        0-* &
         Object attribute as a class instance. \\
    \hline    
        \texttt{ATTRIBUTE} &           
        Any &           
        0-* &
       Object attribute as a simple attribute. \\
    \hline    
        \texttt{COLLECTION} &           
        Any &           
        0-* &
         Object attribute  as a collection.\\
    \hline 
\end{tabulary}
     \caption{Allowed children for \texttt{INSTANCE}} 
     \label{tbl:instance-chilren}
 \end{table}
 
       
\newpage
\textbf{Original}


VO-DML structured types are annotated by using the INSTANCE
element. Note that there is no difference, from a schema point of view,
between \texttt{ObjectTypes} and \texttt{DataType}.


 \begin{lstlisting}[frame=single,caption={INSTANCE child of GLOBALS},style=XML,basicstyle=\tiny]
<dm-mapping:INSTANCE ID="SpaceFrame_ICRS" dmtype="coords:SpaceFrame">
	<dm-mapping:INSTANCE dmrole="coords:SpaceFrame.refPosition"
                                 dmtype="coords:StdRefLocation">
		<dm-mapping:ATTRIBUTE dmrole="coords:StdRefLocation.position" 
		                          dmtype="ivoa:string"  value="NoSet" />
	</dm-mapping:INSTANCE>
	<dm-mapping:ATTRIBUTE dmrole="coords:SpaceFrame.spaceRefFrame" 
	                          dmtype="ivoa:string" value="ICRS" />
	<dm-mapping:ATTRIBUTE dmrole="coords:SpaceFrame.equinox" 
	                          dmtype="coords:Epoch"	value="2015" />
</dm-mapping:INSTANCE>
\end{lstlisting}


\begin{table}[!htbp]
\small
\centering
\begin{tabulary}{\linewidth}{|c|J|}       
       \hline 
            \textbf{Attribute} & 
            \textbf {Role}\\
       \hline         \hline  
            @ID & 
            Element ID, MUST be unique within the mapping block  \\
        \hline 
            @dmrole & 
            INSTANCE role in the DM \\
        \hline 
            @dmtype & 
            Class name \\
        \hline 
     \end{tabulary}
     \caption{\texttt{INSTANCE} attributes} 
     \label{tbl:instance-att}
 \end{table}

\begin{table}[!htbp]
\small
\centering
\begin{tabulary}{\linewidth}{|c|c|c|J|}
    \hline 
        \textbf{@ID} &
        \textbf{@dmrole} &
        \textbf{@dmtype} &
        \textbf{Pattern}\\
    \hline      \hline  
        MAND &           
        NO or EMPTY&           
        MAND &           
        MUST be applied when the  \texttt{INSTANCE} is child of \texttt{GLOBALS}. The element has no role because it is not embedded in a model mapping block. It must be referable by a \texttt{REFERENCE}  \\
    \hline   
        OPT &           
        MAND &           
        MAND &           
        MUST be applied in any other location. It may be referable a \texttt{REFERENCE} . \\
   \hline 
\end{tabulary}
     \caption{Valid attribute patterns for  \texttt{INSTANCE}} 
     \label{tbl:instance-pattern}
 \end{table}


\begin{table}[!htbp]
\small
\centering
\begin{tabulary}{\linewidth}{|c |c |c|J|}
    \hline 
        \textbf{Element} &
        \textbf{Position} &
        \textbf{Cardinality} &
        \\
    \hline      \hline  
        \texttt{PRIMARY\_KEY}  &        
        First &           
        0-* &
        Primary key to be used to in a JOIN context.\\
    \hline    
        \texttt{REFERENCE}  &        
        Any &           
        0-* &
         Object attribute as a reference to either another INSTANCE or a COLLECTION.\\
    \hline    
        \texttt{INSTANCE} &           
        Any &           
        0-* &
         Object attribute as a class instance. \\
    \hline    
        \texttt{ATTRIBUTE} &           
        Any &           
        0-* &
       Object attribute as a simple attribute. \\
    \hline    
        \texttt{COLLECTION} &           
        Any &           
        0-* &
         Object attribute  as a collection.\\
    \hline 
\end{tabulary}
     \caption{Allowed children for \texttt{INSTANCE}} 
     \label{tbl:instance-chilren}
 \end{table}
 
 
