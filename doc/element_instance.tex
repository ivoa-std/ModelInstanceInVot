VO-DML structured types are annotated by using the INSTANCE
element. Note that there is no difference, from a schema point of view,
between \texttt{ObjectTypes} and \texttt{DataType}.

 \begin{lstlisting}[caption={INSTANCE child of GLOBALS},style=XML,basicstyle=\small]
<dm-mapping:INSTANCE ID="SpaceFrame_ICRS" dmtype="coords:SpaceFrame">
	<dm-mapping:INSTANCE dmrole="coords:SpaceFrame.refPosition"
                                 dmtype="coords:StdRefLocation">
		<dm-mapping:ATTRIBUTE dmrole="coords:StdRefLocation.position" 
		                          dmtype="ivoa:string"  value="NoSet" />
	</dm-mapping:INSTANCE>
	<dm-mapping:ATTRIBUTE dmrole="coords:SpaceFrame.spaceRefFrame" 
	                          dmtype="ivoa:string" value="ICRS" />
	<dm-mapping:ATTRIBUTE dmrole="coords:SpaceFrame.equinox" 
	                          dmtype="coords:Epoch"	value="2015" />
</dm-mapping:INSTANCE>
\end{lstlisting}

\begin{table}[!htbp]
\small
\centering
\begin{tabulary}{\linewidth}{|c|J|}       
       \hline 
            \textbf{Attribute} & 
            \textbf {Role}\\
       \hline         \hline  
            @ID & 
            Element ID, MUST be unique within the mapping block  \\
        \hline 
            @dmrole & 
            INSTANCE role in the DM \\
        \hline 
            @dmtype & 
            Class name \\
        \hline 
     \end{tabulary}
     \caption{\texttt{INSTANCE} attributes} 
     \label{tbl:instance-att}
 \end{table}

\begin{table}[!htbp]
\small
\centering
\begin{tabulary}{\linewidth}{|c|c|c|J|}
    \hline 
        \textbf{@ID} &
        \textbf{@dmrole} &
        \textbf{@dmtype} &
        \textbf{Pattern}\\
    \hline      \hline  
        MAND &           
        NO or EMPTY&           
        MAND &           
        MUST be applied when the  \texttt{INSTANCE} is child of \texttt{GLOBALS}. The element has no role because it is not embedded in a model mapping block. It must be referable by a \texttt{REFERENCE}  \\
    \hline   
        OPT &           
        MAND &           
        MAND &           
        MUST be applied in any other location. It may be referable a \texttt{REFERENCE} . \\
   \hline 
\end{tabulary}
     \caption{Valid attribute patterns for  \texttt{INSTANCE}} 
     \label{tbl:instance-pattern}
 \end{table}


\begin{table}[!htbp]
\small
\centering
\begin{tabulary}{\linewidth}{|c |c |c|J|}
    \hline 
        \textbf{Element} &
        \textbf{Position} &
        \textbf{Cardinality} &
        \\
    \hline      \hline  
        \texttt{PRIMARY\_KEY}  &        
        First &           
        0-* &
        Primary key to be used to in a JOIN context.\\
    \hline    
        \texttt{REFERENCE}  &        
        Any &           
        0-* &
         Object attribute as a reference to either another INSTANCE or a COLLECTION.\\
    \hline    
        \texttt{INSTANCE} &           
        Any &           
        0-* &
         Object attribute as a class instance. \\
    \hline    
        \texttt{ATTRIBUTE} &           
        Any &           
        0-* &
       Object attribute as a simple attribute. \\
    \hline    
        \texttt{COLLECTION} &           
        Any &           
        0-* &
         Object attribute  as a collection.\\
    \hline 
\end{tabulary}
     \caption{Allowed children for \texttt{INSTANCE}} 
     \label{tbl:instance-chilren}
 \end{table}
 
 
