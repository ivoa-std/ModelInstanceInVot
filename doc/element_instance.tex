The INSTANCE element defines a complex ObjectType or DataType.
It may be a child of several other elements, and the requirements on
the content (especially dmid and dmrole), may differ depending on
the usage:


\begin{enumerate}
\item Child of \texttt{GLOBALS}

   In this case the INSTANCE is a single stand-alone instance which
   may or may not be referenced by other INSTANCEs. This pattern is typically used for 
   shared object e.g. space frames, or for head object of science product models e.g. Time Series
  \begin{itemize}
     \item May have dmid attribute, as possible target of REFERENCE.ref
     \item must have no  \texttt{dmrole} attribute or an empty one
  \end{itemize}  
     
\item Child of TEMPLATES:
  In this case, the INSTANCE is a template for instances which
  are generated once per row of the associated table.  
  \begin{itemize}
     \item may have dmid attribute, as target of JOIN.dmref
     \item must have no  \texttt{dmrole} attribute or an empty one
  \end{itemize}  

\item Child of COLLECTION:
  There are 2 uses for this pattern.  
  \begin{itemize}
     \item each member INSTANCE is a target for selection using
           the PRIMARY/FOREIGN\_KEY elements. This pattern is only 
           allowed within the GLOBALS environment. In this case:             
           \begin{itemize}
             \item must contain at least one PRIMARY\_KEY sub-element
             \item may have a dmid attribute, as possible target of REFERENCE
             \item must have no  dmrole attribute or an empty one. VODML does set roles to particular collection items
           \end{itemize}

     \item Elements INSTANCE are collection cells with multiplicity > 1
          Each one :             
           \begin{itemize}
             \item must have no dmrole attribute or an empty one
           \end{itemize}
  \end{itemize}  
    
\item Child of INSTANCE: 
     In this case, each INSTANCE represents 
     a complex ObjectType or DataType playing a role in the parent INSTANCE.     
     \begin{itemize}
        \item may have dmid attribute
        \item must have non-empty dmrole attribute
     \end{itemize}
           
\item any INSTANCE:     
   \begin{itemize}
        \item if dmid attribute is present, it must not be empty
        \item must have non-empty dmtype attribute
    \end{itemize}
\end{enumerate}  
    
   
\begin{lstlisting}[frame=single,caption={Example of INSTANCE child of GLOBALS},style=XML,basicstyle=\tiny]
<dm-mapping:INSTANCE dmid="SpaceFrame_ICRS" dmtype="coords:SpaceFrame">
	<dm-mapping:INSTANCE dmrole="coords:SpaceFrame.refPosition"
                                 dmtype="coords:StdRefLocation">
		<dm-mapping:ATTRIBUTE dmrole="coords:StdRefLocation.position" 
		                          dmtype="ivoa:string"  value="NoSet" />
	</dm-mapping:INSTANCE>
	<dm-mapping:ATTRIBUTE dmrole="coords:SpaceFrame.spaceRefFrame" 
	                          dmtype="ivoa:string" value="ICRS" />
	<dm-mapping:ATTRIBUTE dmrole="coords:SpaceFrame.equinox" 
	                          dmtype="coords:Epoch"	value="2015" />
</dm-mapping:INSTANCE>
\end{lstlisting}   
   

\begin{table}[!htbp]
\small
\centering
\begin{tabulary}{\linewidth}{|c|J|}       
       \hline 
            \textbf{Attribute} & 
            \textbf {Role}\\
       \hline         \hline  
            @dmid & 
            Element dmid, MUST be unique within the mapping block  \\
        \hline 
            @dmrole & 
            INSTANCE role played in the DM \\
        \hline 
            @dmtype & 
            Class name in a model context\\
        \hline 
     \end{tabulary}
     \caption{\texttt{INSTANCE} attributes} 
     \label{tbl:instance-att}
 \end{table}   
 


 
\begin{table}[!htbp]
\small
\centering
\begin{tabulary}{\linewidth}{|c |c |c|J|}
    \hline 
        \textbf{Element} &
        \textbf{Position} &
        \textbf{Occurs} &
        \\
    \hline      \hline  
        \texttt{PRIMARY\_KEY}  &        
        First &           
        0-* &
        Primary key to be used to in a JOIN context.\\
    \hline    
        \texttt{REFERENCE}  &        
        Any &           
        0-* &
         Object attribute as a reference to either another INSTANCE or a COLLECTION.\\
    \hline    
        \texttt{INSTANCE} &           
        Any &           
        0-* &
         Object attribute as a class instance. \\
    \hline    
        \texttt{ATTRIBUTE} &           
        Any &           
        0-* &
       Object attribute as a simple attribute. \\
    \hline    
        \texttt{COLLECTION} &           
        Any &           
        0-* &
         Object attribute  as a collection.\\
    \hline 
\end{tabulary}
     \caption{Allowed children for \texttt{INSTANCE}} 
     \label{tbl:instance-chilren}
\end{table}
 
\begin{table}[!htbp]
\small
\centering
\begin{tabulary}{\linewidth}{|c|c|c|J|}
    \hline 
        \textbf{@dmid} &
        \textbf{@dmrole} &
        \textbf{@dmtype} &
        \textbf{Pattern}\\
    \hline      \hline  
        OPT &           
        NO or EMPTY&           
        MAND &           
        MUST be applied when the  \texttt{INSTANCE} is child of \texttt{GLOBALS} or \texttt{TEMPLATES}. The element has no role because it is not embedded in a model component. It must be referable by a \texttt{REFERENCE}  \\
    \hline   
        OPT &           
        MAND &           
        MAND &           
        MUST be applied in any other location. It may be referable a \texttt{REFERENCE} . \\
   \hline 
\end{tabulary}
     \caption{Valid attribute patterns for  \texttt{INSTANCE}} 
     \label{tbl:instance-pattern}
 \end{table}       
\newpage

 
