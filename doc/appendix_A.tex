This listing shows an annotated light curve from Gaia DR2. 
It is based on a data file provided by the Gaia DPAC which mixes photometric points of multiple sources 
through different filters taken all at different time.
This dataset has been reworked out and mapped by hand in a way to provide examples of mapping patterns 
applied to real data.
The mapped models are the prototypes that were used in 2021 to develop the mapping syntax 
and the associated tooling. Models are however inspired by CubeDM, Measure, Coords and PhotDM.
This sample must not be understood as a proposal for serializing light curves but as a demonstrator for annotating complex
data structure.

Most of the XML snippets of the normative sections are taken out from this table. 
In this case, links connecting the text with the relevant locations in the listing have been setup.

The VOTable below contains many comments explaining how mapping patterns must be interpreted.
\lstinputlisting[caption={Gaia multiband example. Notice that due to a Latex tweak, all \_ characters are prefixed with a \textbackslash},language=XML,escapeinside={@}{@}]{appendix_A.xml}

