The aim of the mapping syntax is to enable services to associate a model view to searched data.
Model views can be applied either to legacy data or 
The usage of such model views can range for a simple enhancement of the meta-data up to representation of full model instances.
\begin{itemize}
  \item 
\end{itemize} 


Mapping legacy on a model

data enhance column description

Give the role of column groups

Allows the map complex data structures on legacy data


A step toward a better DM integration in the VO consists in enabling services to annotate
legacy data by providing complete model
views. This requires the server to operate a post-processing inserting into VOTables
annotations that bind data columns with model leaves. .

The mapping syntax allows to map data on any model compliant with VODML. 
These annotations are built as leading XML blocks in VOTables. 
Annotation blocks denote the model structure and contain references to the appropriate table
FIELDs. Model-aware clients can build model instances just by reading the annotation
block and by resolving the FIELD references to get the model leaf values. 

The server must be able to automatically generate such annotations. For this, it
must check that the selected columns match with the model definitions and thus can be
mapped on that model. To operate the mapping, the server needs further information
such as coordinate frames and data profile resources giving the binding between table
columns and model leaves. A prototype (Louys et al. 2021) implementing this feature
has been demonstrated 3.

Allow VOTable to carry complete model instances

TAP services can also be used to host model instances. In this case, we must not
map data on a model anymore but we have to do a real object relational mapping.
However, proposing a common ORM schema is not on the VO roadmap. The work
around strategy is to propose one specific standard per model. This has been done first
for ObsTAP (Tody et al. 2011) which flattens the ObsCore model on one table. This is
also the case for ProvTAP 4 which proposes a relational view for Provenance (Servillat
et al. 2020) data. A prototype (ProvHiPS) tracing the provenance of HST HiPS tiles
has been implemented demonstrated. As the model mapping is defined by a standard,
there is no need to add extra information to the TAP service. Both TAP\_SCHEMA
content and meta-data defined in that standard provide all pieces of information needed
to construct model instances from query results. There are however 2 major issues: 1)
Provenance instances cannot be serialized in one single table; in order to solve this issue
resulting VOTable documents must either contain multiple tables or provide a flattened
view of the model itself (namely last step provenance) 2) The client must be able to tell
the server it is searching Provenance instances.

