The aim of the mapping syntax is to enable services to associate a model view to searched data including legacy data.
The usage of such model views can range for a simple enhancement of the metadata up to the representation of full model instances.
A step toward a better DM integration in the VO consists in enabling services to annotate
legacy data by providing complete model
views. 
To automatically generate model annotations, the server must check that the selected columns 
match the model definitions and thus can be
mapped on that model. To operate the mapping, the server needs further information
such as coordinate frames and data profile resources giving the binding between table
columns and model leaves. A prototype \citep{2201.01732} implementing this feature
has been demonstrated at the ADASS conference in 2021.

This prototype underlined the lack of a query pattern able to tell users which columns must be selected to get a 
complete model view on the table rows. The safest solution to get that view is to query all columns with \texttt{MAXREC=0}.
This would give to the user a quick look on the table columns that are required to build instance of the model(s) of interest.
This could also be handle by a registry extension a mentioned in Appendix E.

