The \texttt{WHERE} element is used to filter iteration outputs. Results are accepted when the key is equal to the specified \texttt{@value}. The mapping syntax does not specify the data types to be used to evaluate the expression. 
There  are 2 different uses for this element:
\begin{enumerate}
\item{As a child of \texttt{TEMPLATES}:}

  Only the table rows satisfying the \texttt{WHERE} conditions will be mapped. 
  With this pattern \texttt{WHERE} must have one \texttt{@primarykey} attribute and one \texttt{@value} attribute. 
  \texttt{@primarykey} references the column (FIELD) to be checked. 
  The \texttt{WHERE} condition is satisfied for the rows having \texttt{@primarykey} equals to \texttt{@value}.
             
\item{As a child of \texttt{JOIN}:}
      
  Only the joined data items satisfying the \texttt{WHERE} conditions will be taken. 
  With this pattern \texttt{WHERE} must have one \texttt{@foreignkey} attribute and one of either \texttt{@value} or \texttt{@primarykey} attribute. 
  \texttt{@foreignkey} references the column of the foreign collection to be checked. 
  The \texttt{WHERE} condition is satisfied for the rows having \texttt{@foreignkey} equals to either \texttt{@value} or \texttt{@primarykey} value.

\end{enumerate}

\begin{lstlisting}[caption={\texttt{WHERE} Example: only rows having val1 as col1 value and  val2 as col2 value are mapped.},language=XML]
<TEMPLATES tableref="table">
  <WHERE primarykey="col1" value="val1" />
  <WHERE primarykey="col2" value="val2" />
  <INSTANCE  dmtype="type" />
</TEMPLATES>
\end{lstlisting}

\begin{lstlisting}[caption={\texttt{WHERE} Example: the join is satisfied when the value of the 'ptc' column is equal to the 'ftc' column of the foreign table. },language=XML]
<JOIN tableref="ft" >
	<WHERE foreignkey="ftc" primarykey="ptc" />
</JOIN>
\end{lstlisting}

\begin{table}[!htbp]
\small
\centering
\begin{tabulary}{\linewidth}{|c|J|}       
       \hline 
            \textbf{Attribute} & 
            \textbf {Role}\\
       \hline         \hline  
            \texttt{@primarykey} &
            FIELD identifier of the primary key column \\
        \hline 
            \texttt{@foreignkey} & 
            FIELD identifier of the foreign key column \\
        \hline 
            \texttt{@value} & 
            Literal value the  \texttt{@primarykey} cell must match with\\
        \hline 
     \end{tabulary}
     \caption{\texttt{WHERE} attributes.} 
     \label{tbl:where-att}
 \end{table}

\begin{table}[!htbp]
\small
\centering
\begin{tabulary}{\linewidth}{|c|c|c|J|}
    \hline 
        \textbf{@primarykey} &
        \textbf{@foreignkey} &
        \textbf{@value} &
        \textbf{Pattern}\\
    \hline      \hline  
        MAND &           
        MAND &           
        NO &           
        2 tables join criteria: \texttt{@primarykey} = \texttt{@foreignkey} \\
    \hline     
        MAND &           
        NO &           
        MAND &           
        Simple join criteria: \texttt{@primarykey} = \texttt{@value} \\
   \hline 
\end{tabulary}
     \caption{Valid attribute patterns for  \texttt{WHERE}.}
     \label{tbl:where-pattern}
\end{table}
