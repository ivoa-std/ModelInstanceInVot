\documentclass[11pt,a4paper]{ivoa}
\input tthdefs

\usepackage{array}
\usepackage{tabulary}  % for nicer tables
\usepackage{calc}
\usepackage{placeins}
\setlength\extrarowheight{2pt}

\newcolumntype{L}{>{\centering\arraybackslash}m{3cm}}

\title{Model Instances in Votables}

% see ivoatexDoc for what group names to use here
\ivoagroup{DM}

\newcommand{\TODO}[1]{%
    \noindent%
    \colorbox{todocolor}{%
            \parbox{0.85\linewidth}{\sffamily \textbf{TODO:}\\
            #1}
    }%
    \vspace{2pt}
}

\newcommand{\note}[1]{%
    \noindent%
    \textcolor{darkgrey}{{\sffamily Note:} \emph{#1}}%
}

\newcommand{\comment}[1]{%
    \noindent%
    \textcolor{red}{{\sffamily Comment:} \emph{#1}}%
}

%%%%%%%%%%%%%%%%%%%%%%%%%%%%%%%%%%%%
% XML syntax coloration and formatting
%
\definecolor{todocolor}{rgb}{1,1,0.8}
\definecolor{darkred}{rgb}{0.6,0,0}
\definecolor{rose}{rgb}{1.0,0.88,0.88}
\definecolor{darkgrey}{rgb}{0.35,0.35,0.35}
\definecolor{gray}{rgb}{0.4,0.4,0.4}
\definecolor{darkblue}{rgb}{0.0,0.0,0.6}
\definecolor{maroon}{rgb}{0.5,0,0}
\definecolor{cyan}{rgb}{0.0,0.6,0.6}

\lstset{
  columns=fullflexible,
  showstringspaces=false,
  commentstyle=\color{gray}\upshape
}

\lstdefinelanguage{XML}
{
  morestring=[b]",
  morestring=[s]{>}{<},
  morecomment=[s]{<?}{?>}, 
  morecomment=[s]{<!--}{-->},
  stringstyle=\color{black},
  identifierstyle=\color{darkblue},
  keywordstyle=\color{maroon},
  morekeywords={ref,utype,dmrole, dmtype, value}% list your attributes here
}

\lstdefinestyle{XML}{
    captionpos=b,
    basicstyle=\small\ttfamily
}

%%%%%%%%%%%%%%%%%%%%%%%%%%%
% Document core
%
\author{François Bonnarel}
\author{Gilles Landais}
\author{Laurent Michel}
\author{Jesus Salgado}
\author{Gerard Lemson}

\editor{Laurent Michel}
\editor{Mark Cresitello Dittmar}


% \previousversion[????URL????]{????Concise Document Label????}
\previousversion{This is the first public release}
       

\begin{document}

\begin{abstract}
Vodml-instance-vot proposes a syntax to map VOTable data on any model serialized in VO-DML.
Vodml-instance-vot annotations are grouped in a single XML block located in the VOTable head. The annotation block allows to easily reconstruct the model structure. It designed in a way that the block can be reused on different data sets in order to facilitate the annotation process.
Vodml-instance-vot is enable to join data from different tables
\end{abstract}


\section*{Acknowledgments}
CDS/TDIG/SourceDM contributors

\section*{Conformance-related definitions}

The words ``MUST'', ``SHALL'', ``SHOULD'', ``MAY'', ``RECOMMENDED'', and
``OPTIONAL'' (in upper or lower case) used in this document are to be
interpreted as described in IETF standard RFC2119 \citep{std:RFC2119}.

The \emph{Virtual Observatory (VO)} is a
general term for a collection of federated resources that can be used
to conduct astronomical research, education, and outreach.
The \href{http://www.ivoa.net}{International
Virtual Observatory Alliance (IVOA)} is a global
collaboration of separately funded projects to develop standards and
infrastructure that enable VO applications.

\pagebreak
\section{Introduction}

The first purpose of a model is to provide, for a particular domain, a formal description of the quantities relevant to that domain and how they relate to each other.
The second purpose is to facilitate the interoperability between  different stakeholders involved in the domain. The interoperability consists in arranging exchanged data 
so that any client can understand it without being concerned with its origin thus allowing the same code to process and compare data coming from different archives.  
In other words, the more faithful is the data representation towards the data model, the better is the interoperability.
The challenge for the VO ecosystem is to design the best way to map various data on standard models while respecting the existing frameworks, protocols and data formats.

We could imagine to develop a new data container specific for that purpose but any solution that ignores the existing assets would be utterly useless and would have no chance of being accepted.
The model mapping solution must be based on VOTables since VOTable  \citep{2019ivoa.spec.1021O} is the standard data container within the VO.
Unfortunately, if the VOTable schema allows a very good description of tabular data, it is not well suited for rendering views of complex data models.
In the case of simple DAL protocols, this limitation has been worked around by specifying the metadata that must be present in the query responses; the model mapping is thereby defined by the standard.

This approach is no longer applicable for complex models such as Cube or Provenance for which the data tree may vary from one instance to another.

The basis on which the standard is designed is threefold 1) the data content is apriori unknown (e.g. TAP response), 2) it has to be mapped on models that are not defined yet and 3) the mapping block must be fully integrated within a VOTable context.
It must also allow to bind native data with a model in a way that model-aware software can see it as model instances while maintaining the possibility to access them in their original serialization.

The VOTable schema supports an XML attribute for this purpose, a UType, that partially meets this function. 
UTypes are path-like strings (\texttt{m:a.b.c}) identifying model leaves. 
They are labels attached to a \texttt{FIELD} or \texttt{PARAM} that tells its role in the
model structure, typically a non cyclic graph. 
As there is no common rule to build UTypes,  each data model (e.g. Characterization, Obscore)  must come with its specific lists of UTypes. 

When used as \texttt{FIELD} attributes, UTypes allow one to connect particular columns to model leaves. This mapping method fits very well with the VOTable schema since it does not require any specific XML elements. 
However, some UTypes features that could complicate the mapping process have been discussed many times:

\begin{itemize}
  \item There is no possibility for one  \texttt{FIELD} or \texttt{PARAM} to play multiple roles.
  \item UTypes are not intended to be parsed, so they only identify which role is played by a \texttt{FIELD}/\texttt{PARAM}, 
  in a restricted context. The same element used in different contexts or models have different UTypes; which hinders interoperability.
  \item UTypes constrain to single-table serializations. They do no allow to join data from different tables. 
  \item UTypes do not support annotating the VOTable content towards multiple models 
  (as TimeSeries or Catalog/Source for instance)
\end{itemize}

The UType approach could have been extended to overcome these limitations, but it has been decided to move forward a solution closer to the \texttt{VODML} \citep{2018ivoa.spec.0910L} concepts. 
\texttt{VODML} is a meta-model that gives a standard way to express VO models as machine-readable XML document.
In \texttt{VODML}, the role of model leaves are no longer given by simple strings like UTypes, but by the local role they play in the model hierarchy.




As a consequence, an annotation mechanism such as MIVOT, based on \texttt{VODML} and preserving the model hierarchy can easily provide data representations faithful to that model.

The basis of MIVOT is to insert into the VOTable, an XML block complying with the 
model structure and containing references to the actual data.
These blocks are designed in such a way that a model-aware client can build  model instances by copying that structure and by resolving the references. 
Other clients can just ignore them and rely as usual on the FIELD/PARAM and GROUP structures.




\subsection{Role within the VO Architecture}

\begin{figure}[h]
\centering

% As of ivoatex 1.2, the architecture diagram is generated by ivoatex in
% SVG; copy ivoatex/archdiag-full.xml to archdiag.xml and throw out
% all lines not relevant to your standard.
% Notes don't generally need this.  If you don't copy archdiag.xml,
% you must remove archdiag.svg from FIGURES in the Makefile.

\includegraphics[width=0.9\textwidth]{role_diagram.pdf}
\caption{Architecture diagram for this document}
\label{fig:archdiag}
\end{figure}

Fig.~\ref{fig:archdiag} shows the role this document plays within
the IVOA architecture \citep{2010ivoa.rept.1123A}.


\pagebreak
\section{Use Cases and Requirements}

\subsection{Use Cases}
The aim of the mapping syntax is to enable services to associate a model view to searched data.
Model views can be applied either to legacy data or 
The usage of such model views can range for a simple enhancement of the meta-data up to representation of full model instances.
\begin{itemize}
  \item 
\end{itemize} 


Mapping legacy on a model

data enhance column description

Give the role of column groups

Allows the map complex data structures on legacy data


A step toward a better DM integration in the VO consists in enabling services to annotate
legacy data by providing complete model
views. This requires the server to operate a post-processing inserting into VOTables
annotations that bind data columns with model leaves. .

The mapping syntax allows to map data on any model compliant with VODML. 
These annotations are built as leading XML blocks in VOTables. 
Annotation blocks denote the model structure and contain references to the appropriate table
FIELDs. Model-aware clients can build model instances just by reading the annotation
block and by resolving the FIELD references to get the model leaf values. 

The server must be able to automatically generate such annotations. For this, it
must check that the selected columns match with the model definitions and thus can be
mapped on that model. To operate the mapping, the server needs further information
such as coordinate frames and data profile resources giving the binding between table
columns and model leaves. A prototype (Louys et al. 2021) implementing this feature
has been demonstrated 3.

Allow VOTable to carry complete model instances

TAP services can also be used to host model instances. In this case, we must not
map data on a model anymore but we have to do a real object relational mapping.
However, proposing a common ORM schema is not on the VO roadmap. The work
around strategy is to propose one specific standard per model. This has been done first
for ObsTAP (Tody et al. 2011) which flattens the ObsCore model on one table. This is
also the case for ProvTAP 4 which proposes a relational view for Provenance (Servillat
et al. 2020) data. A prototype (ProvHiPS) tracing the provenance of HST HiPS tiles
has been implemented demonstrated. As the model mapping is defined by a standard,
there is no need to add extra information to the TAP service. Both TAP\_SCHEMA
content and meta-data defined in that standard provide all pieces of information needed
to construct model instances from query results. There are however 2 major issues: 1)
Provenance instances cannot be serialized in one single table; in order to solve this issue
resulting VOTable documents must either contain multiple tables or provide a flattened
view of the model itself (namely last step provenance) 2) The client must be able to tell
the server it is searching Provenance instances.


\subsection{Requirements}
\begin {itemize}
  \item Shy annotation: The model annotation come in a workflow that works very well for years, this is why the first requirement is to no break any existing stack-holder
  \begin {itemize}
    \item The annotation must not alter the VOTable content
    \item The annotation block must be located in a way it can easily be skipped by non model-aware clients
    \item The vocabulary used for the annotations must not overlap the the VOTable elements (names or attributes)    
    \item The annotation schema must keep independant from the VOTable schéma
  \end {itemize}
  
  \item Schema and validation:
  \begin {itemize}
    \item The annotation schema must keep independent from the VOTable schema
    \item The evolution of the annotation schema must not impact the VOTable schema
    \item The annotation syntax must be validated by standard tools usable in any langage
  \end {itemize}
  
  \item Model agnostic:
  \begin {itemize}
    \item The annotation syntax must be able to map data on any VODML compliant model
    \item The annotation syntax must allow client to use their own strategy to consume mapped data:
      \begin {itemize}
        \item just ignore the syntax
        \item just pick some elements of interest 
        \item just pick model meta-data and process data stream as usual
        \item pick whole model instances
      \end {itemize}
  \end {itemize}
  
  
  
  
\end {itemize}

% use XML formatting for listings
\lstset{language=XML}

\pagebreak
\section{Relation to VOTable}

The data model annotation will reside within the scope of a VOTABLE.
\textit{ Describe which version(s) of VOTable should this be expected to work with? V1.3+ } \newline

\noindent \textbf{Location}

The mapping block:
\begin{itemize}
\item MUST be contained in a VOTable RESOURCE with \texttt{type="meta"}.
\item which MUST be the first child of a RESOURCE with \texttt{type="results"}.
\item there MUST be no more than one mapping block per 'results' RESOURCE.
\end{itemize}

The scope of the mapping block is the whole content of the 'results' RESOURCE. \newline

\noindent \textbf{Namespace}

The \texttt{dm-mapping} name space isolates VOTable elements from mapping elements, and MUST be defined on the mapping block. \newline

\begin{lstlisting}[frame=single,caption={Mapping block in a VOTable},style=XML,basicstyle=\tiny]
<VOTABLE xmlns="http://www.ivoa.net/xml/VOTable/v1.3" 
         xmlns:xsi="http://www.w3.org/2001/XMLSchema-instance" version="1.3">
  <RESOURCE type="results">
    <RESOURCE type="meta">
      <dm-mapping:VODML xmlns:dm-mapping="http://www.ivoa.net/xml/merged-syntax">
        ...
      </dm-mapping:VODML>
    </RESOURCE>
    <TABLE name="myDataTable">
      ....
    </TABLE>
  </RESOURCE>
</VOTABLE>
\end{lstlisting}


\pagebreak
\section{Syntax}

  \begin{figure}[h]
    \begin{center}
      \includegraphics[width=\textwidth]{merged-syntax-summary.png}
      \caption{Annotation Syntax Summary}
      \label{fig:summary}
    \end{center}
  \end{figure}

\pagebreak
\subsection{VODML}
The VODML element is the top level container for the mapping eleements for a single VOTable RESOURCE.

\begin{lstlisting}[frame=single,caption={Example VODML mapping block},style=XML,basicstyle=\tiny]
<dm-mapping:VODML>
  <dm-mapping:MODEL>  ...  </dm-mapping:MODEL>
  <dm-mapping:GLOBALS>  ...  </dm-mapping:GLOBALS>
  <dm-mapping:TEMPLATES>  ...  </dm-mapping:TEMPLATES>
   ...
</dm-mapping:VODML>
\end{lstlisting}

\begin{table}[!htbp]
  \small
  \centering
  \begin{tabulary}{\linewidth}{|c |c |c|}
    \hline 
        \textbf{Element} &
        \textbf{Position} &
        \textbf{Cardinality}\\
    \hline
    \hline  
      \texttt{MODEL} &           
      1 &           
      1-*\\
    \hline    
      \texttt{GLOBALS} &           
      2 &           
      0-*\\
    \hline  
      \texttt{TEMPLATES} &           
      3 &           
      0-*\\
    \hline 
  \end{tabulary}
    \caption{Allowed children for \texttt{VODML}} 
    \label{tbl:vodml-children}
\end{table}



\FloatBarrier
 
\subsection{MODEL}
A VOTable can provide serializations for an arbitrary number of data model
types. In order to declare which models are represented in the file, data
providers must declare them through the \texttt{MODEL} elements.
Only models that are used in the file must be declared. A model is
used if at least one element in the mapping block refer to it. In other terms, only models that define vodml-ids used in the
annotation must be declared.

\begin{lstlisting}[caption={Example \texttt{MODEL} mapping block},language=XML]
<VODML xmlns="http://www.ivoa.net/xml/merged-syntax">
  <MODEL name="sample-ext"
                     url="https://www.myorg.net/models/SampleExt-v1.0.vo-dml.xml" />
  <MODEL name="sample" url="https://www.ivoa.net/xml/DNE/Sample-v1.0.vo-dml.xml" />
  <MODEL name="ivoa"   url="https://www.ivoa.net/xml/VODML/IVOA-v1.vo-dml.xml" />
</VODML>
\end{lstlisting}

\begin{table}[!htbp]
  \small
  \centering
  \begin{tabulary}{\linewidth}{|c|J|}       
    \hline 
         \textbf{Attribute} & 
         \textbf {Role}\\
    \hline
    \hline  
         \texttt{@name}  & 
         Name of the mapped model as declared in the \texttt{VODML} XML model serialization.  This attribute MUST not be empty and forms the prefix used in  \texttt{@dmrole} dmtype tags of elements from that model.  \\
    \hline 
         \texttt{@url} & 
         URL to the vo-dml serialization of the model. If present, this attribute MUST not be empty.\\
    \hline 
  \end{tabulary}
  \caption{\texttt{MODEL} attributes} 
  \label{tbl:model-att}
\end{table}


\begin{table}[!htbp]
  \small
  \centering
  \begin{tabulary}{\linewidth}{|c |c |J|}
    \hline 
        \textbf{@name} &
        \textbf{@url} &
        \textbf{Pattern}\\
    \hline      \hline  
        MAND &           
        OPT &           
        Unique attribute pattern supported by \texttt{MODEL}\\
    \hline 
  \end{tabulary}
  \caption{Valid attribute patterns for  \texttt{MODEL}} 
  \label{tbl:model-pattern}
\end{table}

\FloatBarrier

\subsection{GLOBALS}
\begin{table}[!htbp]
\small
\centering
\begin{tabulary}{\linewidth}{|c |c |c|}
    \hline 
        \textbf{Element} &
        \textbf{Position} &
        \textbf{Cardinality}\\
    \hline      \hline  
        \texttt{INSTANCE}          
        Any &           
        0-*\\
    \hline    
        \texttt{COLLECTION} &           
        Any &           
        0-*\\
    \hline 
\end{tabulary}
     \caption{Allowed children for \texttt{GLOBALS}} 
     \label{tbl:globals-chilren}
 \end{table}
\FloatBarrier

\subsection{TEMPLATES}
The \texttt{TEMPLATES} block defines a template for deriving multiple data model instances,
one for each row of the associated VOTable \texttt{TABLE}.  A subset of the associated
\texttt{TABLE} rows may be selected using the \texttt{WHERE} syntax element.

\begin{lstlisting}[caption={Example of a \texttt{TEMPLATES} block mapping the rows of the table \texttt{Results} (see line~\ref{TEMPLATES_snippet} in Appendix~\ref{appendix_A}).
Each row of the table named \texttt{Results} is be mapped as an instance of VO-DML type \texttt{cube:NDPoint}.
Instances mapping a row do not play any particular role.},language=XML]
<VODML xmlns="http://www.ivoa.net/xml/mivot">
    ...
    <GLOBALS>
    ...
    </GLOBALS>
    <!--
       Mapping of the rows of the table Results
     -->  
    <TEMPLATES tableref="Results">
        <INSTANCE dmid="IDtsIDdata" dmrole="" dmtype="cube:NDPoint">
            <COLLECTION dmrole="cube:NDPoint.observable">
                <INSTANCE dmtype="cube:Observable">
                    <ATTRIBUTE dmrole="cube:DataAxis.dependent" dmtype="ivoa:boolean" value="False"/>
                    ...
                </INSTANCE>
            </COLLECTION>
            ...
        </INSTANCE>
    </TEMPLATES>
  </TEMPLATES>
</VODML>
\end{lstlisting}

\begin{table}[!htbp]
  \small
  \centering
  \begin{tabulary}{\linewidth}{|c|J|}
    \hline 
         \textbf{Attribute} & 
         \textbf {Role}\\
    \hline
    \hline  
         \texttt{@tableref} & 
         ID or name of the mapped VOTable \texttt{TABLE}. ID match takes precedence over name matches when resolving the reference. \\
    \hline 
  \end{tabulary}
  \caption{\texttt{TEMPLATES} attributes.} 
  \label{tbl:templates-att}
\end{table}

\begin{table}[!htbp]
  \small
  \centering
  \begin{tabulary}{\linewidth}{|c|J|}
    \hline 
        \textbf{@tableref} &
        \textbf{Pattern}\\
    \hline
    \hline  
        OPT &           
        If \texttt{@tableref} is not present, \texttt{TEMPLATES} maps the first \texttt{TABLE} of the \texttt{RESOURCE}\\
    \hline 
  \end{tabulary}
  \caption{Valid attribute patterns for  \texttt{TEMPLATES}.} 
  \label{tbl:templates-pattern}
 \end{table}

\begin{table}[!htbp]
  \small
  \centering
  \begin{tabulary}{\linewidth}{|c |c |c|J|}
    \hline 
        \textbf{Element} &
        \textbf{Position} &
        \textbf{Occurs} &
        \\
    \hline
    \hline  
        \texttt{WHERE}  &        
        1 &           
        0-* &
        The mapping applies to rows matching the \texttt{WHERE} condition only\\
    \hline    
        \texttt{INSTANCE} &           
        2 &           
        0-* &
        Mapped instance templates\\
    \hline 
  \end{tabulary}
  \caption{Allowed children for \texttt{TEMPLATES}.} 
  \label{tbl:templates-children}
\end{table}

\FloatBarrier

\subsection{COLLECTION}
    \texttt{COLLECTION} is a container element.  It is used in different contexts, each allowing a limited subset of elements for its content. 
     \texttt{COLLECTION} items must all be of the same type.
    \texttt{COLLECTION} can be populated either by a static list of items (\texttt{INSTANCE}, \texttt{ATTRIBUTE}, ..) or by joining to another \texttt{COLLECTION} by using the \texttt{JOIN} statement.    
    
    \begin{enumerate}
    \item{As child of INSTANCE}
      
      The \texttt{COLLECTION} serves as a container for elements with multiplicity $>$ 1.\\
      Examples of usage in this context would be:
      \begin{itemize}
        \item an array attribute
        \item a reference relation with multiplicity $>$ 1
        \item a composition relation with multiplicity $>$ 1
      \end{itemize}
      
    \item{As child of GLOBALS}
          
      The \texttt{COLLECTION} serves as a proxy for TABLE, grouping common \texttt{INSTANCE}  for selection by PRIMARY \texttt{FOREIGN\_KEY}.
      
      Examples of usage in this context would be:
      \begin{itemize}
        \item a set of photometry filters, which apply to various rows of a photometric data table, based on the value of the 'band' column.
        \item a set of Dataset metadata instances, which apply to various rows of a photometric data table, based on the value of the 'id' column.
      \end{itemize}
          
    \item{As child of COLLECTION}
    
	The collection contains a matrix of  atomic values.
        
    \end{enumerate}
   
\begin{lstlisting}[caption={Example of \texttt{COLLECTION} child of \texttt{INSTANCE}.},language=XML]
<INSTANCE dmtype="model:Thing">
    <COLLECTION dmrole="model:Thing.elems">
        <ATTRIBUTE dmtype="model:Foo" value="100" />
        <ATTRIBUTE dmtype="model:Foo" value="110" />
    </COLLECTION>
</INSTANCE>
\end{lstlisting}   

\begin{lstlisting}[caption={Example of \texttt{COLLECTION} child of \texttt{GLOBALS}.},language=XML]
<GLOBALS>
    <COLLECTION dmid="_filters" >
        <INSTANCE dmtype="model:PhotometryFilter" >
            <PRIMARY_KEY dmtype="ivoa:string" value="RP"/>
            <ATTRIBUTE dmrole="model:PhotometryFilter.name" dmtype="ivoa:string"
                                    value="GAIA/GAIA2r.Grp"/>
        </INSTANCE>
        <INSTANCE dmtype="model:PhotometryFilter" >
            <PRIMARY_KEY dmtype="ivoa:string" value="BP"/>
            <ATTRIBUTE dmrole="model:PhotometryFilter.name" dmtype="ivoa:string"
                                    value="GAIA/GAIA2r.Gbp"/>
        </INSTANCE>
    </COLLECTION>
<GLOBALS>
\end{lstlisting}   

\begin{lstlisting}[caption={Example of \texttt{COLLECTION} populated with a JOIN. See Appendix..~\ref{appen_join}.},language=XML]
<GLOBALS>
    <COLLECTION dmrole="_joined_dataj">
        <JOIN tableref="_someTemplates" dmref="_extInst"/>
    </COLLECTION>
<GLOBALS>
\end{lstlisting}   

\begin{lstlisting}[caption={Example of \texttt{COLLECTION} mapping a 2x2 matrix},language=XML]
<TEMPLATES>
  <INSTANCE dmtype="model:matrix_22">
    <COLLECTION dmrole="model:matrix">
    	<COLLECTION>
        	<ATTRIBUTE dmtype="ivoa:real" ref="field_11"/>
        	<ATTRIBUTE dmtype="ivoa:real" ref="field_12"/>
    	</COLLECTION>
    	<COLLECTION>
        	<ATTRIBUTE dmtype="ivoa:real" ref="field_21"/>
        	<ATTRIBUTE dmtype="ivoa:real" ref="field_22"/>
    	</COLLECTION>
    </COLLECTION>
  </INSTANCE>
<TEMPLATES>
\end{lstlisting}   

\begin{table}[!htbp]
  \small
  \centering
  \begin{tabulary}{\linewidth}{|c|J|}       
    \hline 
         \textbf{Attribute} & 
         \textbf {Role}\\
    \hline
    \hline  
         \texttt{@dmid} & 
         Datamodel element id, MUST be unique within the document.\\
    \hline 
         \texttt{@dmrole} & 
         Role of the \texttt{COLLECTION} in the data model. \\
    \hline 
  \end{tabulary}
  \caption{\texttt{XML attributes for COLLECTION} .} 
  \label{tbl:collection-att}
 \end{table}

\begin{table}[!htbp]
  \small
  \centering
  \begin{tabulary}{\linewidth}{|c|c|c|J|}
    \hline 
      \textbf{Context} &
      \textbf{@dmid} &
      \textbf{@dmrole} &
      \textbf{Pattern}\\
    \hline      \hline  
      1 &
      OPT & 
      MAND & 
      The element maps a collection playing a role in a modeled \texttt{INSTANCE}.  \texttt{@dmrole} MUST NOT be empty.  If present, \texttt{@dmid} MUST NOT be empty. \\
    \hline   
      2 &
      MAND & 
      NO & 
      The collection has no role. It MUST have non-empty  \texttt{@dmid} to reference for ORM selection of contained \texttt{INSTANCE}. This occurs when e.g. the \texttt{COLLECTION}  is child  of  \texttt{GLOBALS}\\
       \hline   
      3 &
      OPT& 
      NO & 
      The element maps a collection within another \texttt{COLLECTION}.  \texttt{@dmrole} MUST not set or empty.  If present, \texttt{@dmid} MUST NOT be empty. \\
    \hline 
  \end{tabulary}
  \caption{Valid XML attribute patterns for \texttt{COLLECTION}. } 
  \label{tbl:collection-pattern}
 \end{table}

\begin{table}[!htbp]
  \small
  \centering
  \begin{tabulary}{\linewidth}{|c|c|J|}
    \hline 
      \multicolumn{3}{|l|}{\textbf{Context: Child of INSTANCE}} \\
    \hline 
      \textbf{Element} & \textbf{Occurs} & \\
    \hline
    \hline  
        \texttt{ATTRIBUTE} & 0-* & Collection of attributes.\\
    \hline    
        \texttt{REFERENCE} & 0-* & Collection of references.\\
    \hline    
        \texttt{INSTANCE} &  0-* &  Collection of instances.\\
    \hline    
        \texttt{JOIN} & 0-1 & Collection populated with a set of joined instances.\\
                 &        & No other child is supported in this case.\\
    \hline    
        \texttt{COLLECTION} & 0-* & Collection of collections.\\
    \hline    
    \hline 
      \multicolumn{3}{|l|}{\textbf{Context: Child of GLOBALS}} \\
    \hline 
      \textbf{Element} &\textbf{Occurs} &  \\
    \hline
    \hline    
        \texttt{INSTANCE} & 0-* & Collection of related instances.\\
    \hline    
        \texttt{REFERENCE} & 0-* & Collection of related references.\\
    \hline  
  \end{tabulary}
     \caption{Allowed children for \texttt{COLLECTION}.} 
     \label{tbl:collection-chilren}
 \end{table}

\FloatBarrier

\subsection{INSTANCE}

\textbf{Mark proposal (as interpreted by LM}

   The INSTANCE element defines a complex ObjectType or DataType.
   
\begin{lstlisting}[frame=single,caption={Example of INSTANCE child of GLOBALS},style=XML,basicstyle=\tiny]
<dm-mapping:INSTANCE ID="SpaceFrame_ICRS" dmtype="coords:SpaceFrame">
	<dm-mapping:INSTANCE dmrole="coords:SpaceFrame.refPosition"
                                 dmtype="coords:StdRefLocation">
		<dm-mapping:ATTRIBUTE dmrole="coords:StdRefLocation.position" 
		                          dmtype="ivoa:string"  value="NoSet" />
	</dm-mapping:INSTANCE>
	<dm-mapping:ATTRIBUTE dmrole="coords:SpaceFrame.spaceRefFrame" 
	                          dmtype="ivoa:string" value="ICRS" />
	<dm-mapping:ATTRIBUTE dmrole="coords:SpaceFrame.equinox" 
	                          dmtype="coords:Epoch"	value="2015" />
</dm-mapping:INSTANCE>
\end{lstlisting}   
   

\begin{table}[!htbp]
\small
\centering
\begin{tabulary}{\linewidth}{|c|J|}       
       \hline 
            \textbf{Attribute} & 
            \textbf {Role}\\
       \hline         \hline  
            @ID & 
            Element ID, MUST be unique within the mapping block  \\
        \hline 
            @dmrole & 
            INSTANCE role in the DM \\
        \hline 
            @dmtype & 
            Class name \\
        \hline 
     \end{tabulary}
     \caption{\texttt{INSTANCE} attributes} 
     \label{tbl:instance-att}
 \end{table}   
    It may be a child of several other elements, and the requirements on
    the content (especially ID and dmrole), may differ depending on
    the usage:
    
\begin{itemize}
\item Child of GLOBALS:
   In this case the INSTANCE is a single stand-alone instance which
   may or may not be referenced by other INSTANCEs.
  \begin{itemize}
     \item must have ID, as possible target of REFERENCE.ref
     \item must have no or empty \texttt{dmrole}
  \end{itemize}  
  
\item Child of TEMPLATES:
  In this case, the INSTANCE is a template for instances which
  are generated once per row of the associated table.  
  \begin{itemize}
     \item may have ID, as target of JOIN.dmref
     \item must have no or empty dmrole \texttt{dmrole}
  \end{itemize}  
  
\item Child of COLLECTION:
  There are 2 uses for this pattern.  
  \begin{itemize}
     \item each member INSTANCE is a target for selection using
           the PRIMARY/FOREIGN\_KEY elements. This pattern is only 
           allowed within the GLOBALS environment. In this case:             
           \begin{itemize}
             \item must contain at least one PRIMARY\_KEY sub-element
             \item must have ID, as possible target of REFERENCE.ref
             \item must have no or empty dmrole
           \end{itemize}

     \item Elements INSTANCE are collection cells with multiplicity > 1
          Each one has:             
           \begin{itemize}
             \item must have ID, as possible target of REFERENCE.ref. 
                   this pattern is             
                   only allowed if within the GLOBALS environment
             \item must have no or empty dmrole
           \end{itemize}
    
     \item Child of INSTANCE: In this case, each INSTANCE represents 
           a complex ObjectType or DataType playing a role in the parent
           INSTANCE.     
           \begin{itemize}
             \item must not have ID (may not be referenced) ??
             \item must have non-empty dmrole
           \end{itemize}
           
     \item any INSTANCE:     
           \begin{itemize}
             \item if ID is present, it must not be empty
             \item must have non-empty dmtype
           \end{itemize}
    \end{itemize}  
  
\end{itemize}  

 
\begin{table}[!htbp]
\small
\centering
\begin{tabulary}{\linewidth}{|c |c |c|J|}
    \hline 
        \textbf{Element} &
        \textbf{Position} &
        \textbf{Cardinality} &
        \\
    \hline      \hline  
        \texttt{PRIMARY\_KEY}  &        
        First &           
        0-* &
        Primary key to be used to in a JOIN context.\\
    \hline    
        \texttt{REFERENCE}  &        
        Any &           
        0-* &
         Object attribute as a reference to either another INSTANCE or a COLLECTION.\\
    \hline    
        \texttt{INSTANCE} &           
        Any &           
        0-* &
         Object attribute as a class instance. \\
    \hline    
        \texttt{ATTRIBUTE} &           
        Any &           
        0-* &
       Object attribute as a simple attribute. \\
    \hline    
        \texttt{COLLECTION} &           
        Any &           
        0-* &
         Object attribute  as a collection.\\
    \hline 
\end{tabulary}
     \caption{Allowed children for \texttt{INSTANCE}} 
     \label{tbl:instance-chilren}
 \end{table}
 
       
\newpage
\textbf{Original}


VO-DML structured types are annotated by using the INSTANCE
element. Note that there is no difference, from a schema point of view,
between \texttt{ObjectTypes} and \texttt{DataType}.


 \begin{lstlisting}[frame=single,caption={INSTANCE child of GLOBALS},style=XML,basicstyle=\tiny]
<dm-mapping:INSTANCE ID="SpaceFrame_ICRS" dmtype="coords:SpaceFrame">
	<dm-mapping:INSTANCE dmrole="coords:SpaceFrame.refPosition"
                                 dmtype="coords:StdRefLocation">
		<dm-mapping:ATTRIBUTE dmrole="coords:StdRefLocation.position" 
		                          dmtype="ivoa:string"  value="NoSet" />
	</dm-mapping:INSTANCE>
	<dm-mapping:ATTRIBUTE dmrole="coords:SpaceFrame.spaceRefFrame" 
	                          dmtype="ivoa:string" value="ICRS" />
	<dm-mapping:ATTRIBUTE dmrole="coords:SpaceFrame.equinox" 
	                          dmtype="coords:Epoch"	value="2015" />
</dm-mapping:INSTANCE>
\end{lstlisting}


\begin{table}[!htbp]
\small
\centering
\begin{tabulary}{\linewidth}{|c|J|}       
       \hline 
            \textbf{Attribute} & 
            \textbf {Role}\\
       \hline         \hline  
            @ID & 
            Element ID, MUST be unique within the mapping block  \\
        \hline 
            @dmrole & 
            INSTANCE role in the DM \\
        \hline 
            @dmtype & 
            Class name \\
        \hline 
     \end{tabulary}
     \caption{\texttt{INSTANCE} attributes} 
     \label{tbl:instance-att}
 \end{table}

\begin{table}[!htbp]
\small
\centering
\begin{tabulary}{\linewidth}{|c|c|c|J|}
    \hline 
        \textbf{@ID} &
        \textbf{@dmrole} &
        \textbf{@dmtype} &
        \textbf{Pattern}\\
    \hline      \hline  
        MAND &           
        NO or EMPTY&           
        MAND &           
        MUST be applied when the  \texttt{INSTANCE} is child of \texttt{GLOBALS}. The element has no role because it is not embedded in a model mapping block. It must be referable by a \texttt{REFERENCE}  \\
    \hline   
        OPT &           
        MAND &           
        MAND &           
        MUST be applied in any other location. It may be referable a \texttt{REFERENCE} . \\
   \hline 
\end{tabulary}
     \caption{Valid attribute patterns for  \texttt{INSTANCE}} 
     \label{tbl:instance-pattern}
 \end{table}


\begin{table}[!htbp]
\small
\centering
\begin{tabulary}{\linewidth}{|c |c |c|J|}
    \hline 
        \textbf{Element} &
        \textbf{Position} &
        \textbf{Cardinality} &
        \\
    \hline      \hline  
        \texttt{PRIMARY\_KEY}  &        
        First &           
        0-* &
        Primary key to be used to in a JOIN context.\\
    \hline    
        \texttt{REFERENCE}  &        
        Any &           
        0-* &
         Object attribute as a reference to either another INSTANCE or a COLLECTION.\\
    \hline    
        \texttt{INSTANCE} &           
        Any &           
        0-* &
         Object attribute as a class instance. \\
    \hline    
        \texttt{ATTRIBUTE} &           
        Any &           
        0-* &
       Object attribute as a simple attribute. \\
    \hline    
        \texttt{COLLECTION} &           
        Any &           
        0-* &
         Object attribute  as a collection.\\
    \hline 
\end{tabulary}
     \caption{Allowed children for \texttt{INSTANCE}} 
     \label{tbl:instance-chilren}
 \end{table}
 
 

\FloatBarrier

\subsection{ATTRIBUTE}

The \texttt{ATTRIBUTE} element defines either a class attribute or a collection item, both set with atomic values.
The requirements on
the content (especially \texttt{@ref} and  \texttt{@dmrole}, may differ depending on
the usage:


\begin{enumerate}
\item Child of \texttt{INSTANCE}

 The \texttt{ATTRIBUTE} can be seen as a class attribute;
    it must have a \texttt{@dmrole} XML attribute.

In this case, the \texttt{ATTRIBUTE} must be specified by:
  \begin{itemize} 
      \item \texttt{@ref} - reference to a VOTable \texttt{PARAM} or \texttt{FIELD}
      \item \texttt{@value} - a literal
      \item  if both are provided; \texttt{@value} serves as the default 
      if the reference cannot be resolved
  \end{itemize}  

  
\item Child of \texttt{COLLECTION}

In this case the host \texttt{COLLECTION} can be seen as a vector and the \texttt{ATTRIBUTE} as one coordinate of the vector. 
It must have  no \texttt{@dmrole} XML attribute or an empty one.

In this case, the \texttt{ATTRIBUTE} must be specified by:
  \begin{itemize} 
      \item \texttt{@ref} - reference to a VOTable \texttt{PARAM} or \texttt{FIELD}
      \item \texttt{@value} - a literal
      \item if both are provided, \texttt{@value} serves as the default if 
      the reference cannot be resolved
  \end{itemize}  
              
\item Any case :

    The \texttt{ATTRIBUTE} must always have a non-empty \texttt{@dmtype} XML attribute.
    An \texttt{ATTRIBUTE} with a \texttt{@ref} pointing on a \texttt{FIELD} must be located in the \texttt{TEMPLATES} element
    mapping the table containing that field. \texttt{@ref} of \texttt{ATTRIBUTE}-s located in the \texttt{GLOBALS} block can only 
    point at \texttt{PARAM} since the \texttt{GLOBALS} element does not map any data table.
    
\end{enumerate}  
 
 MIVOT does not specify the way to handle NULL values. This remains in charge of the client implementation, 
 not of the syntax. The role of syntax is to tell the client how to build model instances. 
 If a \texttt{@ref} points onto a null \texttt{FIELD} value, the client has to set the corresponding model leaf 
 as NULL as well, but the way it do it is very specific to both language and implementation choices.
    
\begin{lstlisting}[caption={Example of an \texttt{ATTRIBUTE} set with either a column reference or a static value (see line~\ref{ATTRIBUTE_snippet} in Appendix~\ref{appendix_A}). If the column reference cannot be resolved, the attribute will be set with its static value.},language=XML]
<INSTANCE dmtype="cube:Observable">
    <ATTRIBUTE dmrole="cube:DataAxis.dependent" dmtype="ivoa:boolean" value="False"/>
    <INSTANCE dmrole="cube:MeasurementAxis.measure" dmtype="meas:Time"
        <INSTANCE dmrole="meas:Measure.coord" dmtype="coords:MJD"
            <ATTRIBUTE dmrole="coords:MJD.date" dmtype="ivoa:real" ref="IDobstime"/>
            <REFERENCE dmrole="coords:Coordinate.coordSys" dmref="IDtimesys"/>
        </INSTANCE>
    </INSTANCE>
</INSTANCE>
\end{lstlisting}  


\begin{table}[!htbp]
\small
\centering
\begin{tabulary}{\linewidth}{|c|J|}       
       \hline 
            \textbf{Attribute} & 
            \textbf {Role}\\
       \hline         \hline  
            \texttt{@dmrole} & 
            Role of the attribute in the DM\\
        \hline 
            \texttt{@dmtype} & 
            Type of the attribute in the DM\\
        \hline 
            \texttt{@ref} & 
            Reference or name of the \texttt{FIELD} or \texttt{PARAM} that has to be used to set the 
            \texttt{ATTRIBUTE} value. In case of duplicate identifiers, which is possible with reference by name, 
            the \texttt{FIELD} reference supercedes the \texttt{PARAM} one. \\
        \hline 
            \texttt{@value}  &
            Default \texttt{ATTRIBUTE} value. This value is taken if there is no 
            \texttt{@ref} attribue or if \texttt{@ref} cannot be resolved.\\
        \hline 
            \texttt{@unit} & 
            \texttt{ATTRIBUTE} unit. Unit applicable to the attribute value which is given 
            by either \texttt{@value} or \texttt{@ref}. \texttt{@unit} must always 
            be compliant with VOUnit \citep{2014ivoa.spec.0523D}. 
            If the attribute value is set by a resolved \texttt{@ref}, 
            \texttt{@unit} must be the VOUnit counterpart of the \texttt{FIELD} or \texttt{PARAM} unit;
            an error must be risen otherwise.\\
        \hline 
            \texttt{@arrayindex} & 
            Index of the native value to be taken to set the \texttt{ATTRIBUTE}. 
            The value must be >= 0.
            Must be ignored if the native value is a single value. 
            An error must be risen if \texttt{@arrayindex} is out of range.
            This attribute is always optional.\\
        \hline 
     \end{tabulary}
     \caption{XML attributes of  \texttt{ATTRIBUTE} .} 
     \label{tbl:attribute-att}
 \end{table}

%mir : I cannot identify clearly the list of usage context . 
%mir: numbers are changed w.r.to the previous list above. 
%mir : should we use a, b, c in the Pattern columns? or ?

\begin{table}[!htbp]
\small
\centering
\begin{tabulary}{\linewidth}{|c|c|c|c|c|J|}
    \hline 
        \textbf{@dmrole} &
        \textbf{@dmtype} &
        \textbf{@ref} &
        \textbf{@value} &
        \textbf{@arrayindex} &
        \textbf{Pattern}\\
    \hline     
    \hline  
        MAND &           
        MAND &           
        OPT &           
        OPT &           
        OPT &   
        (a)\\
    \hline   
        MAND &           
        MAND &           
        NO &           
        MAND &           
        OPT &   
        (b)\\
    \hline  
        NO &           
        MAND &           
        OPT &           
        OPT &           
        OPT &   
        (c) \\
   \hline 
\end{tabulary}
     \caption{Valid attribute patterns for \texttt{ATTRIBUTE}. (a) Valid in a \texttt{TEMPLATES} context.        
        The \texttt{ATTRIBUTE} value must be set with the value of the element referenced by \texttt{@ref}. 
        If the \texttt{@ref} can not be resolved and \texttt{@value} is present, \texttt{@value} must be taken. Either \texttt{@ref} or \texttt{@value} must be present or both. (b) This pattern 
        is valid in any context.  (c) is valid in the context of a \texttt{COLLECTION} item.    
        The \texttt{ATTRIBUTE} value must be set with \texttt{@value} as \texttt{ATTRIBUTE} value.} 
     \label{tbl:attribute-pattern}
 \end{table}

\FloatBarrier

\subsection{REFERENCE}
INSTANCE reference that can be used with the same patterns as for \texttt{INSTANCE} .
There are different uses for the \texttt{REFERENCE} 

\begin{itemize}
    \item Static reference: the element has a \texttt{@dmref} attribute that matches the \texttt{@dmid} attribute of the referenced \texttt{INSTANCE} 
    \item Dynamic reference: The element has a \texttt{@sourceref} attribute identifying  the table where to fetch the referenced column. 
    
             In this case, \texttt{REFERENCE} must be located in a \texttt{TEMPLATES} and it must have one or more \texttt{FOREIGN\_KEY} children. 
             If the referenced table contains several \texttt{INSTANCE} with a \texttt{PRIMARY\_KEY}  the first match must be taken by default.
\end{itemize}

\begin{lstlisting}[caption={Simple \texttt{REFERENCE}, to be replaced with the \texttt{INSTANCE} having \texttt{@dmid} \_tg1 (see \ref{REFERENCE_snippet_1}).},language=XML]
<INSTANCE dmid="IDds1" dmrole="" dmtype="ds:experiment.ObsDataset">
    <ATTRIBUTE dmrole="ds:dataset.Dataset.dataProductType" dmtype="ds:dataset.DataProductType" value="TIMESERIES"/>
    <ATTRIBUTE dmrole="ds:dataset.Dataset.dataProductSubtype" dmtype="ivoa:string" value="GAIA Time Series"/>
    <ATTRIBUTE dmrole="ds:experiment.ObsDataset.calibLevel" dmtype="ivoa:integer" value="1"/>
    <REFERENCE dmrole="ds:experiment.ObsDataset.target" dmref="_tg1"/>
</INSTANCE>

\end{lstlisting}

\begin{lstlisting}[caption={Dynamic \texttt{REFERENCE}, to be replaced with the \texttt{INSTANCE} of the table of collection \_CoordinateSystems and having a \texttt{PRIMARY\_KEY} matching the value of the column  \texttt{\_band}. This pattern is valid in the context of a TEMPLATES (see \ref{REFERENCE_snippet_2}).},language=XML]
<INSTANCE dmrole="meas:Measure.coord" dmtype="coords:PhysicalCoordinate">
    ...
    <!--
        The photometric system is given by the item of the COLLECTION[dmid=IDCoordinateSystems]
        having a primary key matching the value of the column IDband for that particular row
     -->
    <!-- @\label{REFERENCE_snippet}@ back to @\ref{REFERENCE}@ -->
    <REFERENCE dmrole="coords:Coordinate.coordSys" sourceref="_CoordinateSystems">
        <FOREIGN\_KEY ref="_band"/>
    </REFERENCE>
</INSTANCE>
\end{lstlisting}

See more examples in the Appendix \ref{appen_dynref}. 

\begin{table}[!htbp]
\small
\centering
\begin{tabulary}{\linewidth}{|c|J|}       
       \hline 
            \textbf{Attribute} & 
            \textbf {Role}\\
       \hline         \hline  
            \texttt{@dmrole} & 
            Role of the referenced instance or collection in the DM \\
        \hline 
            \texttt{@sourceref}  &
            \texttt{@dmid} of the \texttt{COLLECTION} to be joined with in case of using a \texttt{FOREIGN\_KEY} \\
        \hline 
            \texttt{@dmref} & 
            \texttt{@dmid} of the referenced instance or collection\\
        \hline 
     \end{tabulary}
     \caption{\texttt{REFERENCE} attributes.} 
     \label{tbl:reference-att}
 \end{table}

\begin{table}[!htbp]
\small
\centering
\begin{tabulary}{\linewidth}{|c |c |c|J|}
    \hline 
        \textbf{Element} &
        \textbf{Position} &
        \textbf{Occurs} &
        \\
    \hline      \hline  
        \texttt{FOREIGN\_KEY}  &        
        First &           
        0-* &
        Foreign key to be used to resolve a dynamic reference.\\
    \hline 
\end{tabulary}
     \caption{Allowed children for \texttt{REFERENCE}.} 
     \label{tbl:reference-children}
\end{table}


\begin{table}[!htbp]
\small
\centering
\begin{tabulary}{\linewidth}{|c|c|c|J|}
    \hline 
        \textbf{@dmrole} &
        \textbf{@sourceref} &
        \textbf{@dmref} &
        \textbf{Pattern}\\
    \hline      \hline  
        MAND &           
        MAND &           
        NO &           
        This is the \texttt{FOREIGN\_KEY} pattern \texttt{@sourceref} gives the  \texttt{@dmid} of the \texttt{COLLECTION} to be joined with. In this case \texttt{REFERENCE} must have at least one \texttt{FOREIGN\_KEY} child and the joined \texttt{COLLECTION} must have a \texttt{PRIMARY\_KEY}\\
    \hline   
        MAND &           
        NO &           
        MAND &           
        Simple reference to either an \texttt{INSTANCE} or \texttt{COLLECTION}, usually searched in the \texttt{GLOBALS}\\
   \hline 
\end{tabulary}
     \caption{Valid attribute patterns for  \texttt{REFERENCE}.}
     \label{tbl:reference-pattern}
\end{table}


\FloatBarrier

\subsection{JOIN}
The JOIN element allows to populate a COLLECTION with INSTANCEs from another collection, namely the foreign collection.
The foreign collection can either be a static element (GLOBALS/COLLECTION) or a collection of INSTANCES resulting from the iteration over a TEMPLATES.

The JOIN element must contain attributes identifying the foreign collection  (@sourceref  and/or @dmref). 
It can have WHERE children stating the join condition.

It must be child of a COLLECTION that has no other children element than INSTANCE. A COLLECTION cannot host more than 1 JOIN.

JOIN may have 2 uses:

\begin{itemize}

  \item Join with TEMPLATES data:
       \begin{itemize}
         \item must have a @sourceref attribute identifying the foreign TEMPLATES
         \item may have a @dmref attribute the identify an INSTANCE within the foreign TEMPLATES.
         \item The mapping of the enclosing COLLECTION items can be either implicit or explicit:
         \begin{itemize}
             \item If it has no other children, it must be populated with the whole table rows. In this case the TEMPLATES must have one unique child 
                      since we cannot have collection items made of multiple instances
             \item Otherwise the item content is given by the enclosed INSTANCE which must be unique.
                     The reference (@ref) contained in that INSTANCE must only refer to fields of the foreign table.
         \end{itemize}
  \end{itemize}

       
  \item Join with COLLECTION data:
       \begin{itemize}
         \item must have no @sourceref  attribute, referencing a table in this case is irrelevant.
         \item must have a @dmref attribute the identify the COLLECTION to be joined with. This COLLECTION must be GLOBALS child.
         \item The mapping of the enclosing COLLECTION items can be either implicit or explicit:
         \begin{itemize}
             \item If it has no other children, it must be populated with the foreign collection items.
             \item Otherwise the item content is given by the enclosed INSTANCEs. The reference (@ref) contained 
                      in those INSTANCEs must refer to te field of the foreign table, we must not have such @ref if the foreign 
                      collection is static
       \end{itemize}
  \end{itemize}
           
\end{itemize}

\begin{lstlisting}[frame=single,caption={\texttt{JOIN} },style=XML,basicstyle=\tiny]

\end{lstlisting}


\begin{table}[!htbp]
\small
\centering
\begin{tabulary}{\linewidth}{|c|J|}       
       \hline 
            \textbf{Attribute} & 
            \textbf {Role}\\
       \hline         \hline  
             @sourceref & 
            Reference of the TEMPLATES to be joined with. \\
        \hline 
            @dmref & 
            Reference of the \texttt{COLLECTION} (in \texttt{GLOBALS} to be joined with. \\
        \hline 
     \end{tabulary}
     \caption{\texttt{JOIN} attributes} 
     \label{tbl:join-att}
 \end{table}

\begin{table}[!htbp]
\small
\centering
\begin{tabulary}{\linewidth}{|c|c|J|}
    \hline 
        \textbf{@sourceref } &
        \textbf{@dmref} &
        \textbf{Pattern}\\
    \hline      \hline  
        MAND &           
        NO &           
        The join is done against the TEMPLATES identified by @sourceref  \\  
    \hline   
        NO &           
        MAND &           
        The join is done against the \texttt{COLLECTION} identified by @dmref \\
   \hline 
\end{tabulary}
     \caption{Valid attribute patterns for  \texttt{JOIN}}
     \label{tbl:join-pattern}
\end{table}


\begin{table}[!htbp]
\small
\centering
\begin{tabulary}{\linewidth}{|c |c |c|J|}
    \hline 
        \textbf{Element} &
        \textbf{Position} &
        \textbf{Occurs} &
        \\
    \hline      \hline  
        \texttt{WHERE}  &        
        1 &           
        0-* &
         Join condition\\
    \hline 
\end{tabulary}
     \caption{Allowed children for \texttt{JOIN}} 
     \label{tbl:join-chidlren}
 \end{table}

\FloatBarrier

\subsection{WHERE}
The \texttt{WHERE} element is used to filter iteration outputs. 
Results are accepted when the key is equal to the specified \texttt{@value}. 
\begin{itemize}
    \item The mapping syntax does not specify the data types to be used to evaluate the expression. 
    \item The \texttt{WHERE} statement is false if at least one of the values used for the comparison is NULL.
\end{itemize}

There  are 2 different uses for this element:
\begin{enumerate}
\item{As a child of \texttt{TEMPLATES}:}

  Only the table rows satisfying the \texttt{WHERE} conditions will be mapped. 
  With this pattern \texttt{WHERE} must have one \texttt{@primarykey} attribute and one \texttt{@value} attribute. 
  \texttt{@primarykey} references the column (FIELD) to be checked. 
  The \texttt{WHERE} condition is satisfied for the rows having \texttt{@primarykey} equals to \texttt{@value}.
             
\item{As a child of \texttt{JOIN}:}
      
  Only the joined data items satisfying the \texttt{WHERE} conditions will be taken. 
  With this pattern \texttt{WHERE} must have one \texttt{@foreignkey} attribute and one of either \texttt{@value} or \texttt{@primarykey} attribute. 
  \texttt{@foreignkey} references the column of the foreign collection to be checked. 
  The \texttt{WHERE} condition is satisfied for the rows having \texttt{@foreignkey} equals to either \texttt{@value} or \texttt{@primarykey} value.

\end{enumerate}

\begin{lstlisting}[caption={\texttt{WHERE} Example: only rows having \texttt{val1} as \texttt{col1} value and  \texttt{val2} as \texttt{col2} value must be mapped.},language=XML]
<TEMPLATES tableref="table">
  <WHERE primarykey="col1" value="val1" />
  <WHERE primarykey="col2" value="val2" />
  <INSTANCE  dmtype="type">
  ....
  </INSTANCE>
</TEMPLATES>
\end{lstlisting}

\begin{lstlisting}[caption={\texttt{WHERE} Example: the join is satisfied when the value of the 
                            \texttt{\_pksrcid} column is equal to the \texttt{\_srcid} column of the foreign table 
                            (see line~\ref{WHERE_snippet}). },language=XML]
<JOIN dmref="_ts_data">
    <WHERE foreignkey="_srcid" primarykey="_pksrcid" />
</JOIN>
\end{lstlisting}

\begin{table}[!htbp]
\small
\centering
\begin{tabulary}{\linewidth}{|c|J|}       
       \hline 
            \textbf{Attribute} & 
            \textbf {Role}\\
       \hline         \hline  
            \texttt{@primarykey} &
            FIELD identifier of the primary key column \\
        \hline 
            \texttt{@foreignkey} & 
            FIELD identifier of the foreign key column \\
        \hline 
            \texttt{@value} & 
            Literal value the  \texttt{@primarykey} cell must match with\\
        \hline 
     \end{tabulary}
     \caption{\texttt{WHERE} attributes.} 
     \label{tbl:where-att}
 \end{table}

\begin{table}[!htbp]
\small
\centering
\begin{tabulary}{\linewidth}{|c|c|c|J|}
    \hline 
        \textbf{@primarykey} &
        \textbf{@foreignkey} &
        \textbf{@value} &
        \textbf{Pattern}\\
    \hline      \hline  
        MAND &           
        MAND &           
        NO &           
        2 tables join criteria: \texttt{@primarykey} = \texttt{@foreignkey} \\
    \hline     
        MAND &           
        NO &           
        MAND &           
        Simple join criteria: \texttt{@primarykey} = \texttt{@value} \\
   \hline 
\end{tabulary}
     \caption{Valid attribute patterns for  \texttt{WHERE}.}
     \label{tbl:where-pattern}
\end{table}

\FloatBarrier

\subsection{PRIMARY\_KEY}
\begin{table}[!htbp]
\small
\centering
\begin{tabulary}{\linewidth}{|c|J|}       
       \hline 
            \textbf{Attribute} & 
            \textbf {Role}\\
       \hline         \hline  
            @ref& 
            ID of the FIELD used as primary key \\
        \hline 
            @dmtype & 
            Type of the key \\
        \hline 
            @value & 
            Literal key value. Used when the key relates to a \texttt{COLLECTION in the \texttt{GLOBALS}} \\
        \hline 
     \end{tabulary}
     \caption{\texttt{PRIMARY\_KEY} attributes} 
     \label{tbl:primarykey-att}
 \end{table}

\begin{table}[!htbp]
\small
\centering
\begin{tabulary}{\linewidth}{|c|c|c|J|}
    \hline 
        \textbf{@ref} &
        \textbf{@dmtype} &
        \textbf{@value} &
        \textbf{Pattern}\\
    \hline      \hline  
        MAND &           
        MAND &           
        NO &           
        The FIELD referenced by @ref is a primary key. This pattern is used within a \texttt{TEMPLATES} \\
    \hline     
        NO &           
        MAND &           
        MAND &           
        @value gives the key value. This pattern is used to set a primary key to a \texttt{COLLECTION}\\
   \hline 
\end{tabulary}
     \caption{Valid attribute patterns for  \texttt{PRIMARY\_KEY}}
     \label{tbl:primarykey-pattern}
\end{table}
\FloatBarrier

\subsection{FOREIGN\_KEY}
\begin{table}[!htbp]
\small
\centering
\begin{tabulary}{\linewidth}{|c|J|}       
       \hline 
            \textbf{Attribute} & 
            \textbf {Role}\\
       \hline         \hline  
             @ref& 
             Only used in \texttt{REFERENCE}. Identifier of the FIELD that must  match the primary key of the referenced collection \\
     \hline
     \end{tabulary}
     \caption{\texttt{FOREIGN\_KEY} attributes} 
     \label{tbl:foreignkey-att}
 \end{table}

\FloatBarrier

\pagebreak
\section{Changes from Previous Versions}

No previous versions yet.  
% these would be subsections "Changes from v. WD-..."
% Use itemize environments.


% NOTE: IVOA recommendations must be cited from docrepo rather than ivoabib
% (REC entries there are for legacy documents only)
\bibliography{ivoatex/ivoabib,ivoatex/docrepo,merged-syntax}


\end{document}
